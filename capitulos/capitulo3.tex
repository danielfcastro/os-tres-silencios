\chapter*{The settlements}
\chapterCiting{There is a history in all men's lives.}{William Shakespeare}

For millennia, I wandered the vast lands, an eternal observer of humanity's struggles and triumphs. Countless years slipped through my fingers like grains of sand, as I encountered diverse groups of people scattered across the expanse. In the year 14,000 BCE, fate led me to a modest gathering of no more than twenty individuals, engaged in the timeless pursuits of hunting and fishing. Like a chameleon, I seamlessly blended into their midst, participating in their activities, sharing meals with them, all the while refraining from satiating my hunger with their blood. My presence remained concealed, hidden in plain sight.

\begin{center}
	\includegraphics[scale=0.23]{art/images/landscape/settlement-tiny}
	\label{fig:filler}
\end{center}

As time marched on, I continued my journey, gracefully transitioning from one tribe to another, ensuring that my eternal nature remained a well-kept secret. These small societies thrived, adapting their ways to sustain themselves from the abundance of nature. And I, a nomadic figure, followed the ebb and flow of time, until I finally arrived in a place now known as Jericho, a city whose name echoes through the annals of history. It was around 7500 BCE when I beheld its vibrant streets, teeming with merchants bartering an array of goods—livestock, fish, grains, and spices.

Jericho had flourished into a budding urban center, its population reaching around 2,000 souls. Amidst the hustle and bustle of the city, I bore witness to the birth of conflicts and the perpetration of crimes, observing the intricate tapestry of human existence. In the heart of this thriving metropolis, I chronicled my story on clay tablets, manuscripts, and the very walls that contained the echoes of centuries. Each inscription served as a testament to my immortal journey, capturing the knowledge I amassed and the deeds that shaped my eternal existence. Through these writings, a mosaic of my life emerged, a testament to the enduring legacy I carried.

And so, as the tides of time continued to wash over me, I roamed the Earth, inscribing my experiences in the pages of history. From the distant past to the present, my words bore witness to the unfolding of human civilization, a reminder of my perpetual presence amidst the ever-changing world. The tales etched in clay and carved into stone served as a testament to the everlasting existence of one who walked in the shadows, forever bound to the immortal coil.

\section*{Jericho}
It was around 6500 BCE when my path intersected with that of a young lady who suffered unjustly at the hands of her husband. She possessed a petite yet alluring figure, her curves accentuated by the flowing robes of the time. Her long, lustrous hair cascaded down, styled in intricate braids or twists adorned with ribbons and modest jewelry. Her smooth, radiant complexion held a touch of olive hue, while her delicate features boasted a small nose, full lips, and captivating almond-shaped eyes, enhanced with cosmetic allure. With an air of confidence and sophistication, she carried herself with a flirtatious smile and an enchanting gaze that could captivate any observer.

For nearly five years, I observed her enduring the public humiliation inflicted upon her by her unfaithful and lascivious husband. The neighbors joined in, casting insults and offensive slurs, seeking to demean her dignity. While such occurrences were not uncommon, her unwavering strength and indomitable willpower caught my attention. I found myself admiring her resilience and harboring a growing hatred for her husband. Patiently, I waited and subtly made my presence known to her, a nod or a brief exchange to let her know she was not alone.

Eventually, her husband planned his final blow, accusing her of infidelity and subjecting her to public stoning, leaving her tied to a wooden frame to die a slow and agonizing death. It was then that I intervened. Under the cover of darkness, when the streets were deserted, I found her hanging on the brink of life, deprived of water, food, and care. I gently untied her and caught her frail body in my arms, her neck precariously slumped, her pulse barely discernible. I whisked her away to the safety of my sanctuary, where I tended to her wounds, nursing her back to health with compassion and tenderness. In the process, love blossomed between us, a love borne out of understanding, solace, and acceptance.

Together, we forged a bond unlike any other. I shielded her from the torment of her torturer, offering her the nourishment and sustenance she deserved. I adorned her with garments that befitted her true beauty, surpassing the standards of the city. And one fateful day, I revealed my true nature to her, baring my immortal existence. In that moment, she looked into my eyes and embraced me, not for what I appeared to be, but for the depth of my character and the kindness I had shown her. Tears of blood streamed down my face as I held her tightly, cherishing the connection we had forged, wanting nothing more than to be there for her. And in that moment, she spoke, her voice carrying a newfound strength and resolve, \textbf{Keturah}.

\begin{dialogue}
	\speak{Keturah}"Love me, make me like you for I much desire to live by your side for the eternity."
	
	\medskip
	\direct{I had a clear feeling that for a brief moment my dead heart pumped}	
	
	\speak{cain}"I much want to be by your side as equals but I love you dearly to condemn you with my curse."
	\speak{Keturah}"I chose it gladly, voluntarily, no weight should lay over your shoulders about my choice."

	\medskip	
	\direct{I felt guilty for I was not aware of how could I make someone like me and left her there.}	
\end{dialogue}

Caught off guard by the unexpected turn of events and overwhelmed by a profound sense of loneliness, I withdrew into seclusion, immersing myself in the depths of my emotions. Within the solitude of my chamber, on a night of introspective contemplation, an ethereal presence materialized before me. It took the form of a figure composed of swirling shadows, with piercing blue eyes that seemed to penetrate my very soul. Though bearing a resemblance to an angel, there was an unmistakable otherworldly quality to this enigmatic being.

With a graceful motion, the shadowy figure extended its left hand, palm up, offering me a glimpse into a polished globe fashioned from murky, dark pitch. As I peered into its depths, a silent and haunting vision unfolded before my eyes—an unfolding tragedy. I beheld the people of Seth succumbing to the inexorable grip of death, while I, trapped in an eternal existence, grew ever more isolated and desolate. The scene pierced my heart with an agonizing ache, a realization of the profound price I paid for immortality.

In that moment, the weight of my existence pressed upon me, suffusing my being with an unbearable sorrow. The darkness that enveloped me seemed to mirror the depths of my anguish, as I grappled with the eternal solitude that accompanied my immortal nature. The vision revealed the fleeting and fragile nature of human life, contrasting sharply with the timeless and solitary path I had chosen. The pain I felt became an indelible mark upon my soul, etching a profound understanding of the sacrifices I had made and the heavy burden I bore.

Haunted by the visions within the shadowy globe, I retreated further into the recesses of my own introspection. The knowledge of the impending demise of the people of Seth and the ever-growing weight of my own isolation consumed my thoughts, driving me to question the nature of my existence and the choices I had made. Would I forever be condemned to a life bereft of companionship and connection? Would the echoes of loneliness continue to reverberate through the ages, an eternal reminder of my immortal isolation?

In the face of these profound revelations, I was left grappling with the profound paradox of my existence—a paradox that threatened to engulf me in an abyss of despair.

\begin{dialogue}
	\speak{Rabisu}"I am here as your provider, Cain. You may call me \textbf{Rabisu}, a name given to me by many throughout the ages."
	
	\direct{In a somber tone, I expressed my emotions with a terse and dry speech.}
	
	\speak{Cain}"What is your purpose here, Rabisu, and why do you torment me?"
	\speak{Rabisu}"I have shown you the potential outcome if you choose to remain idle, to let the passage of time diminish your power and hinder your journey of redemption. It is a path paved with failure. However, I offer you an alternative. I can guide you in the act of Embrace, allowing you to choose those worthy individuals who will provide you with spiritual support and accompany you in your eternal existence, shielded from the ravages of time."
	\speak{Cain}"Then teach me, for I cannot bear this burden alone."
\end{dialogue}

He imparted his knowledge to me, revealing the intricacies of the Embrace, and a newfound sense of purpose washed over me, alleviating the heaviness that burdened my heart. And it was on that very night, as if in response to my renewed determination, that the ethereal Wings of Uriel manifested before me.

\begin{dialogue}
	\speak{Uriel}"Hear these words, Caine, for though you possess great power and have been marked by God, understand this truth: any child you sire shall bear the weight of your curse. They shall forever wander in the Land of Nod, fearing the flame and the sun, nourishing themselves solely with blood and consuming nothing but ashes. And because they shall inherit your jealous nature, they shall be plagued by endless strife and discord amongst themselves.  I beseech you, do not condemn those descendants of Adam who strive to walk the path of righteousness. Caine, restrain your fearsome embrace! Let not your progeny be damned, but allow them the chance to find redemption and salvation."
	\speak{Cain}"I have no strength to follow my redemption journey alone.  I need to anone myself but a choice poses in front of me and should I follow alone and fail or curse someone else that might give me strenght to follow the path of Golconda and maybe inspire others?".
	\speak{Uriel}"Caine, son of Adam, son of Eve, I hear your plea and understand the weight upon your shoulders. The path of redemption is a difficult one, and the choice you face is not an easy one to make. It is true that by sharing your curse with another, you may find temporary strength and the potential to inspire others. But remember for now thou are responsible for you and the others you make.
	
	Consider the consequences of your actions, for the curse you bestow upon another may burden their soul and lead them down a dark path. It is important to weigh the potential benefits against the potential harm. Seek guidance from the Eternal and listen to the whispers of your own conscience.
	
	Remember, the journey of redemption is not meant to be walked alone, but that does not mean you should embrace others without careful consideration. Reflect upon your intentions, seek wisdom, and choose the path that aligns with your true purpose and the greater good."
\end{dialogue}

As I stood before Uriel, his words resonated deep within my being. The weight of my power and the burden of my curse weighed heavily upon me. The choice I faced was not an easy one, for it held the potential to shape the lives of others, just as my life had been shaped by the curse I bore.

Uriel's voice carried wisdom and compassion, urging me to consider the consequences of my actions. He reminded me that my descendants, if embraced, would forever wander in the Land of Nod, haunted by fear and nourished solely by blood. Their existence would be marked by strife and discord, a reflection of my own jealous nature. It was a solemn truth that I could not ignore.

Yet, I felt the yearning within me for companionship, for someone to share the journey of redemption. The path ahead seemed daunting, and the strength to walk it alone eluded me. The thought of another by my side, someone who could understand my plight and offer support, was tempting.

Uriel's words held caution, warning me of the burden I would place upon another's soul. To bestow the curse of vampirism upon another was a weighty responsibility, one that could lead them down a dark and treacherous path. It was a decision that required careful consideration and reflection.

In that moment, I realized that the journey of redemption was not meant to be undertaken in isolation, but it was a path that demanded utmost discernment and moral conviction. I resolved to seek guidance from the Eternal and to listen to the whispers of my own conscience. I would reflect upon my intentions and weigh the potential benefits against the potential harm.

The fate of those I would embrace hung in the balance, and their souls deserved nothing less than my earnest contemplation. I vowed to choose the path that aligned with my true purpose and the greater good, ensuring that my decision would bring hope and inspiration rather than darkness and despair.

With determination in my heart and the weight of my choices upon my shoulders, I embarked on a quest for redemption, mindful of the consequences and prepared to face the challenges that lay ahead.

For over a millennium, we basked in the embrace of love and understanding. I imparted to her every facet of her new nature, guiding her through the intricacies of our existence, sharing the do's and don'ts that would shape her path. But as time went on, \textbf{Keturah} yearned to explore the world on her own, to immerse herself in its wonders and to know its diverse inhabitants. Her beauty and allure were matched only by her insatiable curiosity.

With a heavy heart, I granted her the freedom to venture forth, just as I once had. The fear of losing her gripped me tightly, for the uncertainty of never seeing her again cast a shadow over my soul.

\section*{Urkadash}
During my time in Urkadash, a smaller yet well-established trading hub, I had the opportunity to witness the potential of a remarkable individual named \textbf{Naram-Sin}. He was a dedicated trader who harbored deep concerns for the growth and prosperity of the town. His unwavering commitment to Urkadash struck a chord within me, as I shared his vision of transforming the settlement into a beacon that would attract people from far and wide. I was moved by his selflessness, and I made the decision to bestow the embrace upon him.

Little did I know at the time that the embrace itself was a curse, a direct consequence of my encounter with the angels of the Lord. This revelation served as a stark reminder that I could not take this gift lightly, for it came with great responsibility.

The Divine One above also mentioned a path called Golconda, which held the promise of forgiveness for my sins. As a proud man, I understood that true self-forgiveness must be earned and that I must embark on a journey of personal redemption before treading this path. One way I believed I could atone for my transgressions was by carefully selecting those individuals who proved themselves worthy and who worked tirelessly in the service of mankind to receive the embrace. \textbf{Naram-Sin} was one such individual, and I hoped that his future deeds would help me in my quest for redemption.

\textbf{Naram-Sin}, a merchant hailing from the region we now know as Babylon, possessed an array of exceptional qualities. His commanding presence, combined with his tall and lean physique, featured a sharp jawline and piercing dark eyes that seemed to miss nothing. He often adorned himself in fine linen garments of various colors, including a long skirt and a belted tunic, paired with meticulously crafted leather sandals. It was evident that he had mastered the art of haggling and negotiation, skills I personally witnessed during our interactions. His charming and persuasive way of speaking added to his appeal, making him a formidable figure in the world of trade.

But there was more to \textbf{Naram-Sin} than met the eye. He carried an air of mystery and intrigue, setting him apart from his peers. Rumors circulated about his connections with influential and wealthy patrons, as well as his possession of rare and exotic goods from distant lands. Regardless of the truth behind these rumors, he exuded an air of confidence and promise, establishing himself as a rising star in the bustling realm of Babylonian commerce.

As I continued to observe \textbf{Naram-Sin}, I discovered that he was a man of his word. Since his arrival in Urkadash, the settlement began to garner attention from other traders, thanks to his efforts. Recognizing his potential and leadership qualities, I chose to embrace him, believing that he possessed the ability to propel the settlement's growth and development. True to my expectations, the population of Urkadash burgeoned from a mere 2,000 to over 10,000 within a span of 15 years. However, as the population grew, so did the problems that plagued the town.

Suddenly, \textbf{Naram-Sin} found himself thrust into the spotlight, attracting unwanted attention. Awakened humans, as they referred to themselves, arrived to conduct research or so they claimed. Soon, more of their kind followed suit, exacerbating the situation. While it would have been simple for me to eliminate them, I understood that such an act would only be a temporary solution. It would neither address the underlying problem nor contribute to my pursuit of atonement. I explained my perspective to \textbf{Naram-Sin}, and although he grasped the concept, he did not experience the same anguish as I did. Nevertheless, he chose to confront the threats head-on, engaging directly with the wild beasts that encroached upon the town. In doing so, he learned, albeit through the harshest of lessons, the consequences of his actions. Tragically, his family fell victim to the ensuing conflict, plunging his life into despair.

After several decades of introspection and seeking closure through self-discovery, \textbf{Naram-Sin} made the heart-wrenching decision to leave Urkadash. Feeling a deep connection to him, I opted to accompany him on his journey, at least for a time. We ventured together, sharing in the trials and tribulations of our immortal existence, but as the road unfolded before us, we eventually found ourselves compelled to take separate paths.

As I watched \textbf{Naram-Sin} walk away, my heart heavy with a mix of sorrow and gratitude, I couldn't help but wonder if our paths would ever cross again. The bond we had forged during our time in Urkadash remained strong within me, and the memories of our shared experiences served as a constant reminder of the complex nature of the choices I faced. It was a constant reminder of the weight I carried and the importance of selecting worthy individuals for the embrace.

Leaving Urkadash behind, I set forth on a new chapter of my eternal existence, driven by a desire to fulfill my quest for redemption. The road ahead was uncertain, and the challenges I would encounter were vast, but I knew deep within me that the choices I made and the individuals I embraced would shape not only my own destiny but also the fate of those who walked alongside me. And so, with a heavy heart yet a glimmer of hope, I ventured into the unknown, ready to face the trials that awaited me on my journey to Golconda.

\section*{The Path of Caine}
After parting ways with \textbf{Naram-Sin}, I found myself alone, a powerful being yet plagued by a sense of helplessness. The recent setbacks I had faced forced me to reevaluate my actions and contemplate whether true reconciliation and forgiveness with God were truly what I desired. It seemed that every attempt I made to right my wrongs only resulted in further missteps, burdening me with an ever-increasing weight upon my soul.

In the midst of this existential turmoil, I turned inward, embarking on a journey of self-reflection and introspection. I withdrew from the world, forsaking nourishment, abstaining from speech, and relinquishing all concerns for my own well-being. This was a pivotal moment, for I knew that succumbing to my nightmares, my inner demons, and my insatiable Beast would render me too fragile to forgive myself, rendering true atonement impossible.

For forty days and forty nights, I sought solace in the depths of caves, seeking the company of animals and facing the shadows that lurked within. The walls of the cave became a canvas upon which my inner struggles were projected, my fears manifesting as haunting apparitions that relentlessly assailed my mind, body, and soul. Maintaining my sanity became a Herculean task, as shadows emerged from every corner, and the line between reality and illusion blurred. In that moment of vulnerability, my \textbf{Beast} emerged, a primal force unleashed by my own inner turmoil. I was besieged by illusions and consumed by an inferno raging within me, my voice silenced, and my very essence assaulted by the monstrous manifestations of my own fears.

Amidst the chaos, a vision appeared in my dreams—an image of a corrupted and crumbling tower, a symbol of my own shattered self. From within that darkened structure emerged a figure clad in white robes, his curly brown hair falling gently upon his shoulders. It was my brother, a beloved presence that my fractured spirit yearned to behold. Though I could barely bring myself to meet his gaze, his kind eyes radiated a profound sense of peace. In his presence, my restless spirit found solace, and the tempest within me began to calm.

Awakening from this nightmarish vision, I found my body covered in blood, as though I had perspired it from my very pores. But I also emerged with a newfound understanding—a revelation that would guide my path towards self-redemption. From the depths of my shattered being, I forged a set of tenets, principles to anchor me in the turbulent sea of my existence. Each tenet served as a guiding light, reminding me of the path I must walk to retain control over myself and navigate the treacherous terrain of my immortal life.

Regret washed over me, realizing that if only I had possessed this wisdom before, I could have imparted it to my children, sparing them from the pain and unpleasantness that accompanied their existence. But now, armed with this hard-earned knowledge, I resolved to honor their memory and strive for a future where the mistakes of the past would not be repeated. It was a solemn commitment, an oath I swore to myself as I embarked on the next phase of my eternal journey.

\begin{itemize}
	\item Restrain yourself and never give in to extremes
	\item Develop your Willpower
	\item Accept your condition and that you are no longer human
	\item If possible, leave the children of Seth alone and coexist with them
\end{itemize}


\section*{Back to Jericho}
As I continued my search for a place to stay in Jericho, my path crossed with \textbf{Zillah}. His vision of beauty stirred my heart and I felt drawn to his enchanting presence. However, something seems off. \textbf{Zillah} seems indifferent, as if she has no feelings, no soul. I'm not the height of beauty, but I know how to woo a woman, and my attempts to woo her have been met with indifference. Yes, we know that women can sometimes be dismissive in order to value themselves, but this shows signs. With her it was different, it wasn't repulsion, it wasn't disdain, it wasn't like she asked to try harder, it was like she was never there, the physical presence yes, but an empty container, without an inner self, almost as if there was no soul. This sparked my curiosity and I longed to understand the reason behind it.

In the midst of my confusion, an unsettling feeling began to weigh upon me. It was as if an ominous presence lurked in the shadows, watching our every move. \textbf{Zillah} too seemed to sense this unseen threat, and fear gripped her heart, prompting her to flee from my presence.

I could not ignore the sense of foreboding that washed over me. It was unlike any monitoring or observation I had experienced before. This presence felt sinister, devoid of any specific origin or purpose. Its purpose was unclear, but one thing was certain—it posed a threat to Zillah and possibly to all of Jericho.

Determined to uncover the truth behind this mysterious presence, I allowed my instincts to guide me. It led me to a deserted construction site, where I took refuge from the scorching sun. It was there that I sensed the gaze upon me once again. I looked around and found myself face to face with a group of individuals, eight in total.

Introducing themselves as the Taftâni, the protectors of Jericho, they acknowledged their awareness of my presence and the unique qualities I possessed. It was a relief to discover that the magnificent city of Jericho was indeed under the watchful eye of guardians who understood the importance of safeguarding its prosperity and inhabitants.

In a peaceful exchange, we shared our stories, and they revealed their mission to protect Jericho from various threats, including powerful wizards and djinn associated with chaos and destruction, whose nickname was The Crone. The Crone was seen wandering the outskirts of the city, sometimes even walking by the streets at night, but her place, her home, was never actually found. They also revealed the existence of sealed demons spread throughout the city, whose influence posed a danger to all who lived within its boundaries. The Taftâni had lost many of their comrades while investigating these malevolent forces and their fear was palpable. Maybe Te Crone could be looking for these demons, maybe just trying to wreak havoc, destabilize the city, maybe setting the stage for something bigger, or she could just be walking around without much interest in that specific place.  Anyways it was a topic to be investigated.

Recognizing the gravity of the situation, I suggested that communication and understanding were key among us in order to solve the findings presented to me. A constant communication or monitoring was pivotal for success. After a series of amiable conversations, we bid each other farewell, choosing to embark on our respective paths.

Despite my desire to be with Zillah, my attempts to win her affection proved futile. Her aloofness wounded me deeply, leaving a void in my heart. Little did I know that this icy distance was a result of her own fear, sensing the looming threat that followed us.

As the shadowy presence grew more prominent, Zillah's apprehension reached its peak, causing her to flee from my side. I was left alone, plagued by both the absence of her love and the looming threat that seemed to chase her. The task ahead was clear—I had to uncover the source of this ominous presence, not only to protect Zillah but also to safeguard the city that had become a beacon of hope in my eternal journey.

\begin{center}
	\includegraphics[scale=0.24]{art/images/filler/hammurabi_codex}
	\label{fig:filler}
\end{center}
\section*{The Crone}
\begin{center}
	\includegraphics[scale=0.21]{art/images/The Crone}
	\label{fig:filler}
\end{center}
Determined to protect Zillah and the city of Jericho, I found myself delving deeper into my own skills and knowledge. As I continued my studies I felt the desire to walk at night, and indeed it was a beautiful landscape with full light and a dark sky clear of clouds. In my wanderings outside the walls of Jericho I came across a mysterious figure. an old woman dressed in rags, with gray, almost white hair who walked slightly bent over. That was really strange, how a fragile lady like that would have the courage to wander incautiously through places where bandits are known to lurk, not to mention other creatures I've had the displeasure of coming across. I looked a second time, more curious, attentive, and like an arrow hitting my head I was struck by the recognition that that lady was actually The Crone.

Intrigued by the old woman's enigmatic presence, I couldn't help but feel drawn to her, as if an aura of fascination drew me in. Little did I know that her fascination was a carefully crafted web designed to ensnare innocent souls like mine. The Crone, aware of my vulnerability and thirst for redemption, saw an opportunity to exploit my desires and manipulate me to serve her twisted agenda. But until then I didn't know anything about it, nothing other than that she had a good chance of being The Crone. And I need to play that game, pretend I didn't know who she was.

\begin{dialogue}
	\speak{Cain}"State your business, for my mind is burdened with thoughts."
	\speak{Lady}"I am but a wanderer in search of knowledge and fortune."
	\speak{Cain}"I cannot assist you with either, may your journey be safe."
	\speak{Lady}"And to you as well, Cain, the one with a broken heart."
	\speak{Cain}"How do you know me, and who are you exactly?"
	\speak{Lady}"I am well acquainted with your story, for you are an intriguing subject of study. My name is inconsequential, but I am known as The Crone."
	\speak{Cain}"Crone, I dislike being pursued and watched. Attend to your own affairs."
	\speak{The Crone}"And yet, you did the same with the lady you hold dear. A broken heart is a tragic thing, and I can sense your sorrow, as she shows no interest in you."
	\speak{Cain}"I have heard of your immense power and knowledge in the mystical arts."
	\speak{The Crone}"You flatter me. I perceive that you desire something. Perhaps we could converse in private, for there are eyes searching for me, and there are already eyes upon you."
\end{dialogue}

We walked along the road, eventually veering onto a narrow path that led us to his house. A large portico adorned with enigmatic inscriptions greeted us and we crossed a short passage paved with translucent stones, flanked by lush vineyards. As we entered his residence, I noticed that it seemed more spacious than its exterior suggested. The atmosphere inside emanated a serene tranquility that enveloped me.

When my gaze fell on her, I saw her true form. The weight of the years was evident on her face, her eyes lacking the brightness of youth. She had a full set of teeth and, despite her apparent age, her arms exuded surprising strength. However, I noticed a slight limp in the left leg, an indication of the passage of time and the difficulties faced. Her skin bore the effects of age, looking dry and slightly aged.

\begin{dialogue}
	\speak{The Crone}"Rest peacefully from the sun's embrace, my guest, and tomorrow we shall delve into our discussion."
	\speak{Cain}"I appreciate your hospitality, Crone, but my stay must be brief. I have pressing matters that require my attention."
\end{dialogue}

I found myself in a dimly lit chamber, devoid of windows, nestled in a sumptuous bed. Upon awakening, I discovered The Crone engrossed in reading ancient manuscripts. Sensing my presence, she beckoned me closer.
\begin{dialogue}
	\speak{The Crone}"Do not fear, Cain. Come closer," she said in a soothing tone.
	\speak{Cain}"Why are you showing me such kindness, Crone?"
	\speak{The Crone}"We must extend compassion to those who yearn for what they cannot possess, for their pain can be more burdensome than meets the eye. Easing their burden even slightly can bring solace."
	\speak{Cain}"Very well, then. You seem perceptive, yet lacking complete understanding. You can perceive the shadow but lack the details of the object that casts it, oh Crone. Do you believe your wisdom can aid me?"
	\speak{The Crone}"Indeed, if you divulge to me what ails you, I can consult the Moon and the Stars, seek the wisdom of the gods, and uncover a solution for your pain. And perhaps, in return, you can offer guidance for my own tribulations."
	\speak{Cain}"I long for the love of Zillah, even though her heart holds no space for me."
	
	\direct{The first drink}
	\speak{The Crone}"My blood will reveal the answers we seek. Drink from me now, for I sense your hunger. However, do not indulge excessively, for I must seek the solution to your enigma. Stay, peruse my texts, rest, and tomorrow the visions regarding your query will become clearer. Then, I shall begin my work."
\end{dialogue}

I awoke and roamed through The Crone's house, searching for her presence. Peering out the window, I beheld a sight that surprised me — she lay semi-naked upon a stone, seemingly collapsed. Her ceremonial garments adorned her body, intentionally revealing her breasts. Hurriedly, I approached and gently covered her exposed bosom with her robe, then guided her back inside.

\begin{dialogue}
	\speak{The Crone}"I thank you for your kindness and apologize for the manner in which you found me."
	\speak{Cain}"There is no need for apologies, Crone. I am relieved to see you safe."
	\speak{The Crone}"I beseeched the divine, communed with those above, and obtained a partial solution. I now understand that a tender heart can be reached through the right impressions. You must touch her soul."
	\speak{Cain}"But I cannot transform rocks into silk. There is no human way to accomplish such a feat."
	\direct{The second drink}
	\speak{The Crone}"That is precisely why you sought my aid, Cain. I possess a potion that can fulfill your desires. Drink from my blood, and by tomorrow, I shall provide what you seek."
\end{dialogue}

As I drank from her, The Crone fell into a deep slumber. I approached the ancient scrolls she had left behind and continued my study. Eventually, I succumbed to exhaustion and slept. The next day, I discovered The Crone waiting for me in the very spot where I had found her the previous day. She requested a clay bowl and a ceremonial knife.

\begin{dialogue}
	\speak{The Crone}"I hold the answer to your predicament."
	\direct{She gathered a handful of berries she had collected beforehand, crushed them into the bowl, extended her right arm, and instructed me to make a cut and pour her blood into the vessel.}
	\direct{The third drink}
	\speak{Cain}"Drink, oh Cain, master of the Beast, dweller of Darkness."
\end{dialogue}

As I consumed the contents of the bowl, a surge of confidence and power coursed through my veins, filling me with the belief that I could now win over the heart of my beloved Zillah. But as I finished drinking, a sinister laughter erupted from The Crone, resonating deeply within the environment. It was in that moment that I realized I had been deceived, betrayed by her cunning manipulation. Fury burned within me, and I instinctively reached out to seize her neck, but my hand remained motionless, paralyzed by her unseen force. She commanded me to halt, asserting her control over me.

With a mixture of anger and despair, I understood the true power that the bond of blood could wield, binding individuals together in ways they never intended. The Crone used this power to her advantage, manipulating me to love her against my will. And in an act of ultimate dominance, she commanded me to grant her immortality, compelling me to comply.

In my weakened state, under her compelling influence, I fulfilled her command, bestowing upon her the immortality she desired. The weight of my actions settled upon me, the knowledge that I had succumbed to her treachery and played a part in her eternal existence. I was ensnared in her web, a prisoner to her whims and desires. The consequences of my choices unfolded before me, and I could only hope for a chance to break free from this unholy bond.

As I left The Crone's lair, frustration and regret gnawed at me for falling into her trap so easily. How foolish I had been to trust her and succumb to her manipulations. However, as the initial waves of despair washed over me, I found solace in the knowledge I had acquired through my exploration of her writings and artifacts. I knew the true nature of her deception and saw an opportunity to turn the tables.

With a newfound determination, I sought out Zillah, armed with gifts and tokens of affection. I presented her with a beautiful dress, a bouquet of flowers, and a delicate bottle of perfume. To my surprise, she smiled, genuinely delighted by my gestures, and reciprocated my love and affection. In her presence, I could momentarily forget the hold that The Crone had over me. The love between Zillah and me was genuine, but I could not deny the inexplicable passion that still stirred within me for The Crone.

Though I played the role of a devoted lover to Zillah, my heart remained conflicted. The bond The Crone had forged through her blood magic tethered me to her, creating an unnatural desire that I couldn't shake. My love for Zillah was real, but my allegiance to The Crone was a constant presence, threatening to unravel the happiness I found with Zillah.

As days passed, I lived a double life, divided between my love for Zillah and my servitude to The Crone. It was a delicate balancing act, and the weight of my secret burdened me. I knew that eventually, I would have to face the consequences of my choices and find a way to break free from The Crone's grasp. The path ahead was uncertain, but I was determined to find a solution that would set me free from this unholy bond and allow me to embrace the love I had for Zillah without the shadow of The Crone looming over us.

For a year and a day, I served The Crone faithfully, outwardly obeying her every command while secretly plotting my escape from her clutches. It was a treacherous path I walked, but I was determined to sever the Blood Bond that held me captive. Studying and  delving deeper into the ancient texts and mystical lore, I sought the guidance of the Taftâni, revered masters of enchantments.

In a very discreet way I met with the Taftâni and explained my situation. So they understood, but it would be unfair if I claimed they were comfortable with it, yet they accepted the risk and let me frequent their quarters and study ways to break my connection. Throughout the back and forth between the Crone's lair and Taftâni's quarters, I presented the essence of the rare herb I had obtained on my travels. With reverence and caution, they examined the potent ingredient, recognizing its power to break the bonds that bound me. Together, we embarked on a journey to create a concoction infused with the essence of the herb – a drink that would tear me apart and break any remaining ties.

Under the watchful eyes of the Taftâni, I assisted in the intricate process of brewing the potion. Their hands moved with precision, blending the essence with other mystical ingredients, channeling their ancient knowledge and weaving potent incantations into the mixture. As the brew simmered in the alchemical vessel, a palpable aura of liberation filled the air, signaling the potency of the enchantment at work.

With the concoction prepared, I awaited the Taftâni's guidance on its usage. They shared sacred rituals and invocations that would fortify its effects and ensure its success. Together, we performed the enchantment, infusing the potion with an unbreakable resolve and imbuing it with the power to sever any lingering bonds.

With gratitude in my heart, I raised the chalice to my lips and consumed the enchanted elixir—a liquid that now held the essence of the Taftâni's craft. Its taste, both bitter and sweet, represented the sacrifices made and the hope of liberation that lay within. As the potion coursed through my veins, I felt a surge of energy, an affirmation that the bonds that once held me captive were weakening, their influence waning.

In that transformative moment, I knew that I had become unbondable. No longer would any supernatural chains hold sway over me. The Blood Bond that once bound me to The Crone was shattered, its remnants scattered to the winds. I was free.

Yet, caution prevailed in my heart, for I knew The Crone's keen senses would soon detect the shift in power. I concealed the remaining elixir, safeguarding it as a shield against future threats, should any lingering bonds attempt to ensnare me.

Empowered by the Taftâni's enchantment, I continued to play the role of The Crone's loyal servant, concealing my newfound liberation, and secretly seeking a way to sever all ties with her completely. With the Taftâni's knowledge ingrained within me, I navigated the intricate web of deception, always one step ahead, waiting for the opportune moment to break free from her grasp once and for all.

The path to freedom was arduous, but armed with the power of the elixir, the enchantment of Taftâni and the resilience of my spirit, I was prepared to face any challenges that awaited me. Time was passing and my destiny awaited me. Free from supernatural chains, I embraced the journey that would lead me to final liberation.

With a heavy heart and cautious determination, I approached The Crone one last time, aware of her distracted state and the need to carefully navigate our conversation. I spoke to her, choosing my words with precision, about the betrayal that had befallen me at the hands of my first child. I expressed my deep desire to make amends and seek a way to rectify the situation.

With the newfound strength and resolve, I approached The Crone for one final encounter, concealing my true intentions beneath a facade of remorse. I spoke to her with a somber tone, recounting a tale of betrayal by one of my own offspring—a child who had strayed from the righteous path and caused great harm. I confessed my deep yearning to make amends and seek a way to rectify the situation.

The Crone, seemingly lost in her own thoughts, appeared somewhat absent-minded as I recounted the tale of betrayal. Her attention wavered, and it was clear that my words did not penetrate as deeply as I had hoped. Nevertheless, I persevered, conveying my yearning for forgiveness and the urgency to find a solution.

As The Crone's attention wavered and her demeanor remained distracted, she suddenly issued a command, her voice carrying a hint of urgency. She instructed me to procure a sturdy piece of wood, one that possessed the strength and sharpness akin to that of a stake.

With a sense of curiosity mingled with caution, I ventured into the surrounding woods in search of the specified material. Guided by an instinctual intuition, I came across a tree with robust branches and dense wood, ideal for the task at hand. Carefully selecting a branch, I assessed its weight and texture, ensuring it possessed the necessary qualities to serve as a makeshift stake.

Returning to The Crone's dwelling, I set about the task of sharpening the wood with meticulous precision. Each stroke of the blade against the rough bark was imbued with intention, as I honed the wood to a fine point.  The air filled with the scent of fresh wood shavings as I worked diligently, following The Crone's command.

Once the stake-like object was complete, I presented it to The Crone, who acknowledged my efforts with a distracted nod. Her attention, however, remained elsewhere, her mind seemingly preoccupied by matters unbeknownst to me. The significance of the stake and its purpose remained a mystery, buried beneath her distracted demeanor.

In a moment of unexpected opportunity, The Crone revealed her dark suggestion, urging me to employ the sharpened stake against the heart of those who had wronged me. 

Without hesitation, I swiftly turned the tables on The Crone, plunging the stake into her form with a determined hand piercing her flesh and heart.  As the wood pierced her flesh, a surge of energy crackled through the air, while she was spitting blood her eyes tried to meet mine but she could not for I never would look to her eyes again. The Crone's once commanding presence crumbled into paralysis. She lay motionless, trapped by her own design.

I had four hours before the sunrise and used that to craft more stakes and a base made of wood to stick her there,  I carefully plucked out her eyes globes and placed them on the top of her nipples. The stakes remaining were used to impaling and pinning the wrists and ankles to the wood base I crafted. Laid her there and let the sun do the job.  It was about one hour for the sunrise when the Taftâni appeared before us.

Their expressions mirrored a mix of astonishment and concern as they witnessed The Crone, now rendered powerless, lying vulnerable before us. The air crackled with anticipation as we stood united, united against the manipulative sorceress who had ensnared me in her web.

The Taftâni, possessing ancient wisdom and insight, approached cautiously, assessing the situation with a discerning eye. Zillah, ever the beacon of compassion and love, stood by my side, her presence offering confort in this turbulent moment.

Together, we deliberated on the course of action, aware that our decision held the power to reshape our lives and the lives of others entangled in this web of deception. The Taftâni, drawing upon their knowledge of mystical arts, suggested a ritual to sever the remaining bonds (if any and just for safety for no one would know whatever The Crone might have done without our knowledge) that tethered me to The Crone's influence.

With practiced precision, they initiated a series of incantations and rituals, infusing the air with an otherworldly energy. The words flowed like a gentle river, carrying with them the promise of liberation and renewal. The very essence of the herb that had rendered me unbondable was invoked, amplifying its protective properties.

As the ritual reached its crescendo, I felt a surge of energy coursing through me, like a river breaking free from its dam. The remaining fragments of The Crone's influence shattered, dissipating into the ether, leaving me unburdened and free.

In that transformative moment, a sense of clarity and purpose washed over me. I knew that the path ahead would not be without challenges, but with Zillah's unwavering love and the guidance of the Taftâni, I was no longer bound by the manipulative ties of The Crone.
\begin{dialogue}
	\speak{Cain}"There she lies, The Crone. I beg of you, do not release her. Stay by my side and witness the spectacle as the sun ascends. Let the Archangel Raphael bestow his final kiss upon her."
	\speak{Taftâni}"Remain with us, for I possess the means to shield you from the sun's rays, albeit temporarily. Will you stay and witness the sunrise, if I offer you my protection for the duration of a meal?"
	\speak{Cain}"How unexpected yet enticing. I graciously accept your offer. I shall stay and bear witness to the dawn."
\end{dialogue}

We stood there, gazing at the spectacle unfolding before us, the sun's rays engulfing The Crone in a fiery inferno of vibrant red and brilliant blue. In the midst of my fear, I felt a reassuring grip on my hand—it was Zillah, offering her support in this unsettling moment. As The Crone erupted into a mesmerizing display of flames, we hastily departed from the scene, seeking solace in the safety of my dwelling.

The following day (or night), as the day light fades away, I awoke to find Zillah patiently awaiting my awakening. Her presence filled the room with a comforting warmth and a glimmer of hope. She explained as follows:

"I diligently carried out my mission alongside the Taftâni, constantly on the lookout for any signs of potential danger. It was during one of our surveillance operations that we spotted The Crone making her way to Jericho, uncertain of the threat she might pose. As the primary agent, I took it upon myself to initiate contact and gather information, hoping to uncover her intentions.

Little did I know that this encounter with The Crone would dramatically alter the course of my life. Our interactions, initially driven by necessity, gradually developed into a complex relationship. Unbeknownst to me, The Crone, in her cunning ways, cast a curse upon me—a curse that transformed my heart into stone, bestowed upon me an irresistible visage, and rendered me incapable of forming meaningful connections with others.

This isolation, paradoxically, had an unintended effect. The mysterious allure of my condition attracted numerous individuals, including yourself. Despite the barriers imposed by the curse, you were inexplicably drawn to me, intrigued by the enigma that surrounded my existence."

In our journey together, we traversed the ancient lands surrounding Babylon, exploring numerous cities that rose and fell throughout history. From the bustling streets of Jericho to the enigmatic ruins of Nalakar, Jalmara, Gornak, and Azuruk, we witnessed the rise and demise of civilizations shaped by riots, invasions, and the passage of time.

Our bond grew stronger as they ventured through the remnants of Karziban, Maribas, Arjuna, Varnia, Zolqar, Kharzuk, and Falur, immersing ourselves in the history and culture of each place. Together, we unraveled the mysteries of forgotten civilizations, delving into the forgotten knowledge of ancient times.

Throughout our travels, one city stood the test of time, unyielding to the ebb and flow of history—Ubar. This ancient city, steeped in legend and myth, retained its presence amidst the ever-changing landscape. In Ubar, they found a sense of home, belonging, knowing that amidst the vanishing cities, they had discovered a place where they could carve out a lasting legacy.
\begin{center}
	\includegraphics[scale=1.80]{art/images/filler/obelisk}
	\label{fig:filler}
\end{center}