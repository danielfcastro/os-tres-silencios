\chapter*{Character Sheet Adjustments}
\section*{Traits}
\subsection*{Generation}
In the early epochs of Vampiric History, the concept of Generation background takes on a nuanced significance. When creating a character using this book, it is essential to consider the character without Generation as anything other than fifth generation.\\
 
\begin{itemize}[label={}]
	\item \EmptyDot 5th Generation - This, however, isn't a favorable trait due to the risk it poses, not solely based on blood potency but more critically due to the potential violation of rules. If a character possesses fifth-generation status without proper knowledge or discretion, it might trigger a blood hunt against both the character and their maker.
	\item \FilledDot 4th Generation - Such vampires are recognized entities, subject to constant scrutiny by their creators. They are products of creation following the established laws and traditions of the Cainite world.
	\item \FilledDot \FilledDot 3rd Generation - This status introduces complexity as records of the actual number of 3rd generation shares are unavailable. While the possibility of vampires beyond the original founders of the clans exists, it becomes imperative to meticulously define their creator, the disciplines they possess, their potential lineage, and how Second Generation vampires perceive them. Despite being a neonate, the character faces the gaze of more powerful fourth-generation vampires, who might covet their blood. Secrecy becomes paramount, as opportunistic individuals might exploit any vulnerability.
	\item \FilledDot \FilledDot \FilledDot 2nd Generation - These vampires are as mysterious, if not more so, than their third-generation counterparts. While some vampires of this generation may be known within their circles, the inherent potential attracts attention and greed. The character, essentially a neophyte, must tread carefully, for if discovered, they could become the target of unscrupulous vampires intrigued by their potent blood. Considering the historical backdrop of vampire uprisings against their progenitors, surviving unscathed and maintaining secrets within this environment is a daunting task. The character stands at the intersection of mystery, potential danger, and the delicate dance of keeping their true nature concealed.
\end{itemize}

\subsection*{Library}
Some vampires possess vast collections of occult knowledge, such a collection of tomes or tablets. These volumes may have been gifts from a vampire's mentor or acquired before their Embrace.

These occult repositories prove invaluable for both mundane and mystical research. While True Magick remains beyond the grasp of simple study, understanding the principles behind the vampiric Disciplines is attainable. The resources need not be limited to scrolls, could have access to tablets as well or even special type of contacts that would have occult ou mystical lore, provided the character can study continuously and access the information when needed.

Similar to Mentors, these Libraries aid in spending experience points. Rolling the Library rating (difficulty 7) when acquiring new Knowledge or enhancing Disciplines allows the vampire to save experience points. Each success translates to a point saved, with a minimum expenditure of one point. Research typically spans at least one week, and the character can roll only once per experience expenditure.

Vampires often store their occult volumes at communal havens or other secure locations, freely sharing them within their coterie. Like Chantry, this Background can be pooled if the means allow and the characters remain in close association.\\
\begin{itemize}[label={}]
	\item \FilledDot A collection of common lore of fairy tales and doubious legends, very superficial.
	\item \FilledDot \FilledDot A collection of lore and gossips from sects, superficial but can offer some insight.
	\item \FilledDot \FilledDot \FilledDot A handful of rare and ancient lore, and vast mundane resources.
	\item \FilledDot \FilledDot \FilledDot \FilledDot An impressive collection of occult and mortal lore.
	\item \FilledDot \FilledDot \FilledDot \FilledDot \FilledDot A hoard of lost secrets, a sea of common wisdom.
\end{itemize}

\subsection*{Resources}
Since the money was not invented by the time of the game it will be assumed that for the sake of gameplay the currency would be a primitive type of shekels.\\

\begin{itemize}[label={}]
	\item \FilledDot Small savings: You reside in a modest dwelling within the city's confines and may own a few basic possessions. If all your assets were liquidated, you could gather around 10 shekels. Your monthly allowance is limited to 5 shekels. This status allows you to present yourself as a common member of society, indulging occasionally in gifts and small luxuries suitable for someone of slightly higher standing. You may hire occasional help or a servant, and a fraction of your resources is in portable items, like jewelry, that enables you to maintain a basic standard of living for up to six months.
	\item \FilledDot \FilledDot Middle class: You enjoy a comfortable residence, likely an adequate apartment or house, and possess a means of transportation, perhaps a donkey or a shared cart. If all assets were liquidated, you could amass 80 shekels. Your monthly allowance is around 12 shekels. You project the image of a respected middle-class individual, occasionally indulging in gifts and treats that befit a person of higher standing. Maintaining a servant or hiring specific help is within your means. A portion of your wealth exists in tangible forms, allowing you to sustain a one-dot quality of living for up to six months, regardless of your location.
	\item \FilledDot \FilledDot \FilledDot Large savings: You are a property owner with substantial equity, residing in a prominent dwelling with land. If liquidated, your assets would amount to at least 500 shekels. Your monthly allowance is approximately 30 shekels. You are a distinguished and well-established member of your community, known for your landownership. Your reputation allows you generous credit terms. Most of your wealth is tied up in equity and property, with only a fraction in ready cash. Maintaining a one-dot quality of existence is easily achievable wherever you are, for as long as you desire.
	\item \FilledDot \FilledDot \FilledDot \FilledDot Quite well off: You own an expansive estate, perhaps a considerable house or a somewhat worn country manor. If all assets were liquidated, you would have 5000 shekels. Your monthly allowance reaches 90 shekels. You seldom handle cash, as most of your wealth exists in valuable tangible forms. Compared to your local peers, you are considerably wealthier. When not focused on Resource management, you can comfortably sustain a three-dot existence for up to a year and a two-dot existence indefinitely.
	\item \FilledDot \FilledDot \FilledDot \FilledDot \FilledDot Ridiculously wealthy: You are a multimillionaire, with assets totaling around 50,000 shekels if liquidated. Your monthly allowance is a lavish 300 shekels. You are the epitome of affluence, a model for others to aspire to in popular perception. Your opulent lifestyle garners attention from various media. You possess vast and diversified assets, potentially connected to the fortunes of empires, each with extensive staff and connections throughout society. Your travel accommodations are at least three-dot comforts, and with a bit of effort, even more luxurious. Corporations and regional governments may approach you for business purpose.
\end{itemize}