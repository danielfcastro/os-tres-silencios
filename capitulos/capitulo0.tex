\chapter*{Introduction}
\chapterCiting{Do not go where the path may lead, go instead where there is no path and leave a trail}{Ralph Waldo Emerson}
This book was written inspired by some desire to bring closer to the playable universe what could have happened in time immemorial. Moments that would bypass the biblical origin of humanity that relates to the origin of the vampire according to the "White-Wolf" tradition.

There is obviously no record that things went this way for humanity and much of this book was inspired by apocryphal books that recount the history of humanity, the first family and their destinies.

Some mathematical extrapolation done from front to back was necessary to try to place the possible origin in time but it is just a way of positioning everything.

There is some authorial freedom to talk about the origins of vampires since "White-Wolf" purposely produces content that contradicts each other and this makes everything even more enjoyable to read and feel the mystery of these times and how each person tells their part of the story. Examples of this are the names of the Antidiluvians, which, as they were all from the time of Babylon, made no sense for them to have names unrelated to that region. Another example is the presence of other members of the second generation who try to somehow reconcile reports from other clans with this topic.

I believe these nuances in the telling of the story are interesting for building an atmosphere for a playable environment without hindering the creative freedom of the storyteller.

\section*{What are Babylon times}
In the ancient annals of human history, a pivotal epoch unfolds in the cradle of civilization, a time predating the Classical Era, known as the Babylonian times. Situated in the region of Babylon, this era marks the nascent stages of human existence, a period when the land was still untouched, and humanity wandered in the nomadic tapestry of its infancy. The canvas of Babylonian history begins with vast expanses of untamed wilderness, devoid of settlements or proto-cities. As the narrative unfolds, the trajectory of civilization takes a transformative turn, giving rise to settlements that would later burgeon into the foundational structures of proto-cities.

Much like the later Classical Era, the Babylonian times witnessed the evolution of human societies on multiple fronts. While the Classical Age saw the emergence of great thinkers, philosophers, and powerful empires around the Mediterranean, the Babylonian times lay the groundwork for the very concept of settled communities. It is in Babylon that the early threads of civilization were woven, setting the stage for the intricate tapestry of human history.

In this primordial period, humanity's journey takes root in the fertile grounds of Babylon, ultimately laying the foundation for the rise of settlements and proto-cities. As the nomadic way of life gradually gives way to a more structured existence, the Babylonian times become a crucible for the forging of essential elements that would shape the destiny of civilizations to come. From the emptiness of the ancient landscape, the stage is set for the emergence of settlements that would stand as testament to the resilience and adaptability of humankind in its relentless pursuit of progress.

This era, although preceding the illustrious Classical Age, carries its own unique significance as the cradle of humanity, where the seeds of civilization took root and began to flourish. Through the lens of the Babylonian times, we embark on a journey back to the origins of settled existence, exploring the untamed landscapes that bore witness to the dawn of human civilization and the eventual rise of proto-cities that would define the course of history.

\section*{When this book is located}
In the shadowy tapestry of ancient history, our story unfolds during the sprawling empire of Babylon in the year 4700 BC, a period marked by the consolidation of city-states and the rise of powerful empires. At this juncture, the landscape is teeming with intrigue and political machinations, where the fate of mortal society intertwines with the enigmatic presence of vampires and other supernatural beings.

The stage is set against the backdrop of Babylonian city-states, each vying for dominance and influence. Among them, the second city, renowned as Enoch, emerges as a prominent player, a beacon of civilization amidst the ancient realms. As the narrative unfolds, the players find themselves entangled in the intricate web of governance, where mortal societies strive for power and supremacy.

Circa 4700 BC, this era witnesses the zenith of Babylonian civilization, with successful empires flourishing and city-states engaging in complex political maneuvers. The consolidation of power sets the tone for a realm where mortal leaders navigate the intricacies of governance, facing challenges not only from within their societies but also from the cryptic machinations of vampires and other supernatural entities.

Amid the grandeur of Babylon, whispers of political intrigue and power struggles reverberate through the city-states. Ambitious rulers seek to solidify their dominion, while behind closed doors, supernatural beings weave their own plots, manipulating mortal affairs for their clandestine purposes. The convergence of mortal and supernatural elements creates a volatile concoction, adding layers of complexity to the political landscape of Enoch and its surrounding city-states.

As the empires of this ancient time reach their zenith, the seeds of future conflicts and alliances are sown. In the year 4700 BC, Babylon stands as a testament to the resilience and ambition of humanity, with mortal leaders navigating the challenges of governance, all while supernatural forces lurk in the shadows, shaping the destiny of civilizations. The stage is set for a tale where the lines between mortal and supernatural become blurred, and the actions of leaders resonate through the annals of history, echoing the profound transformations of an era on the cusp of greatness.  That is when the playable part already takes place.  Machinations can occur either on the part from mortal society, vampires and other supernatural beings to derail the city from a glorious path, to usurp power or just wars could erupt, even other supernatural beings could consider cainites as unworthy, unclean, a corruption embodied and well, this might call attention of those willing to do something about.

The reader is encoraged of course to use the book how he seems fit, including a chronicle during the first city if desired.