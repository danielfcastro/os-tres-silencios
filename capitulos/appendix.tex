\chapter*{Appendix}
\section*{Dates}
\begin{itemize}
	\item From expulsion of Cain until meet Lilith took 5664 moons (close to 472 years) and Cain met her in the year 14528 BCE.
	\item It took almost a thousand years to learn all disciplines in full dedication (13528 BCE) when Cain left Lilith.
	\item Cain enters Jericoh for the first time - 6500 BCE
	\item Cain enters Urkadash for the first time - 5500 BCE
	\item Cain enters Jericoh for the second time - 5248 BCE	
	\item Cain enters Ubar for the first time - 5124 BCE	
	\item Cain enters Ubar for the second time, with Zillah time - 4853 BCE	
	\item Cain name Ubar after Enoch - 4853 BCE
	\item The fall of Enoch - 3202 BCE
	\item The rise of the Second City - 2998 BCE
	\item The fall of the Second City - 2387 BCE
\end{itemize}

\onecolumn\sloppy
\section*{Progenies}
It is noteworthy that the vast majority of Antediluvian names, the supposed ones, are probably not the real names, as they are names associated with civilizations that did not yet exist. Thus, the royal names probably relate to the region of Babylon, Assyria, Judea, Cappadocia, Sumer.
\begin{table}[ht]
\centering
\sffamily
\begin{tabular}{|l|c|c|l|l|r|}
	\rowcolor{black!75}
	\hline
	\head{Name} & \head{Generation} & \head{Sex} & \head{Embrace} & \head{Whereabouts} & \head{Later referenced as}\\
	\hline
	Keturah & 2 & Female & 6500 BCE & Unknown & N/A\\
	\hline
	 Naram-Sin & 2 & Male & 5250 BCE & Unknown & N/A \\
	\hline
	Zillah & 2 & Female & 5247 BCE & Unknown & N/A\\
	\hline
	Irad & 2 & Male & 4832 BCE & Unknown & N/A\\
	\hline
	Enoch & 2 & Male & 4847 BCE & Unknown & N/A\\
	\hline
	Shulgi & 2 & Male & 4832 BCE & Unknown & The Lovers\\
	\hline
	Anahita & 2 & Female & 4832 BCE & Unknown & The Lovers\\
	\hline
	Enheduanna & 2 & Female & 4829 BCE & Unknown & N/A\\
	\hline
	Ereshkigal & 2 & Female & 4825 BCE & Unknown & N/A\\
	\hline
	Malik & 2 & Male & 4743 BCE & Unknown & N/A\\
	\hline
	Haqim & 3 & Male & 4705 BCE & Unknown & Assamite\\
	\hline
	Ashurbanipal & 3 & Female & 4811 BCE & Unknown & Brujah\\	
	\hline
	Ashur & 3 & Male & 4822 BCE & Unknown & Cappadocian\\
	\hline
	Ennoia & 3 & Female & 4812 BCE & Unknown & Gangrel\\
	\hline
	Shulgi & 3 & Male & 4807 BCE & Unknown & Lasombra\\
	\hline
	Elisha & 3 & Male & 4801 BCE & Unknown & Malkavian\\
	\hline
	Shamash & 3 & Male & 4811 BCE & Unknown & Nosferatu\\	
	\hline
	Zaid & 3 & Male & 4812 BCE & Unknown & Ravnos\\	
	\hline
	Saulot & 3 & Male & 4830 BCE & Unknown & Salubri\\
	\hline
	Seth & 3 & Male & 4827 BCE & Unknown & Seth\\
	\hline
	Arikel & 3 & Female & 4801 BCE & Unknown & Toreador\\
	\hline
	Marduk & 3 & Male & 4830 BCE & Unknown & Tzimisce\\		
	\hline
	Omar & 3 & Male & 4827 BCE & Unknown & Ventrue\\
	\hline	
\end{tabular}
\end{table}

\begin{center}
	\includegraphics[scale=0.18]{art/images/filler/pngegg}
\end{center}

\section*{Pictures}
\begin{figure}[h]
	\includegraphics[scale=0.42]{art/images/Enoch/Enoch}
	\captionsetup{justification=raggedright,singlelinecheck=false}
	\caption{Enoch from above}
	\label{fig:enoch-above}	
\end{figure}

\twocolumn\sloppy
\section*{Pantheon}
The Babylonian pantheon is renowned for its vastness, comprising a multitude of deities and intricate mythological narratives. The sheer magnitude of the Babylonian pantheon can be attributed to a combination of factors deeply rooted in the cultural and historical context of ancient Mesopotamia.

Firstly, the Babylonian civilization, which thrived in the fertile lands between the Tigris and Euphrates rivers, was a product of diverse cultural interactions and influences. As a result, the Babylonian pantheon incorporates deities from various regional traditions, assimilating gods and goddesses from earlier Sumerian, Akkadian, and Amorite cultures. This amalgamation of different religious beliefs contributed to the rich tapestry of the Babylonian pantheon.

Furthermore, the Babylonian people attributed immense importance to the natural forces and phenomena that governed their daily lives. They believed that every aspect of the world, from celestial bodies and natural elements to societal structures and individual experiences, was under the watchful eye and influence of specific deities. Thus, numerous gods and goddesses were assigned to govern and embody various aspects of the cosmos, including the sun, moon, earth, water, fertility, justice, war, and wisdom, among others.

Additionally, the Babylonians' religious practices involved a complex system of rituals, prayers, and offerings aimed at appeasing and seeking favor from the deities. Each deity had their own sphere of influence and specific rituals associated with their worship. This led to the proliferation of specialized cults and temples dedicated to specific gods and goddesses, thereby expanding the pantheon.

Moreover, the Babylonian pantheon reflected the social and political structures of the civilization. Deities often held close associations with kingship, serving as patrons and protectors of the ruling monarchs. The presence of numerous deities allowed the Babylonian rulers to legitimize their authority by establishing divine connections and claiming divine support for their rule.

Overall, the extensive nature of the Babylonian pantheon can be attributed to a combination of cultural assimilation, reverence for natural forces, religious rituals, and political dynamics. The diverse and complex tapestry of gods and goddesses within the Babylonian pantheon illustrates the profound significance of religion and mythology in ancient Mesopotamia, shaping every facet of the civilization's beliefs, practices, and worldview.
\newpage 
\textbf{{\Large Abgal }}
\begin{itemize}
	\item \textbf{Definition}: God of divine sages who were sent to Earth by Enki in the early times to bestow sacred laws of civilization upon humanity.
	\item \textbf{Civilization}: Associated with Sumerian, Akkadian, and Babylonian civilizations.
	\item \textbf{Classification}: Major gods in Sumerian, Akkadian, and Babylonian mythology.
	\item \textbf{Other Names}: Known as Apkallu or Fish Apkallu by the Akkadians and Babylonians. Also depicted as griffins or winged human-like figures.
\end{itemize}

\textbf{{\Large Absu }}
\begin{itemize}
	\item \textbf{Definition}: God of freshwater and the freshwater depths of the world.
	\item \textbf{Civilization}: Associated with Babylonian, Akkadian, and Sumerian civilizations.
	\item \textbf{Classification}: Major god in Babylonian, Akkadian, and Sumerian mythology.
	\item \textbf{Other Names}: Also known as Apsu and Abzu.
\end{itemize}

\textbf{{\Large Adad }}
\begin{itemize}
	\item \textbf{Definition}: God of storms and a darker version of the Sumerian god Ninurta.
	\item \textbf{Civilization}: Associated with Babylonian and Sumerian civilizations.
	\item \textbf{Classification}: Major god in Babylonian and Sumerian mythology.
	\item \textbf{Other Names}: Known as Ishkur by the Sumerians.
\end{itemize}

\textbf{{\Large Adramelech }}
\begin{itemize}
	\item \textbf{Definition}: God of the sun in Babylonian mythology and personification of the sun, provider and sustainer of life.
	\item \textbf{Civilization}: Associated with Babylonian culture.
	\item \textbf{Classification}: Major god in Babylonian mythology.
	\item \textbf{Other Names}: No known alternative names.
\end{itemize}

\textbf{{\Large Aja }}
\begin{itemize}
	\item \textbf{Definition}: Goddess of dawn and associated with youth, sexual love, and marriage.
	\item \textbf{Civilization}: Associated with the Akkadian and Sumerian civilizations.
	\item \textbf{Classification}: Major goddess in Akkadian and Sumerian mythology.
	\item \textbf{Other Names}: Also known as Aya or Sherida in older Sumerian tradition.
\end{itemize}

\textbf{{\Large Apkallu Griffin}}
\begin{itemize}
	\item \textbf{Definition}: God of Babylonian version of the Abgal.
	\item \textbf{Civilization}: Associated with Babylonian culture.
	\item \textbf{Classification}: Major figure in Babylonian mythology.
	\item \textbf{Other Names}: No known alternative names.
\end{itemize}

\textbf{{\Large Apsu }}
\begin{itemize}
	\item \textbf{Definition}: God of freshwater and the primordial sweet waters that surrounded the earth.
	\item \textbf{Civilization}: Associated with Babylonian, Akkadian, and Sumerian civilizations.
	\item \textbf{Classification}: Major god in Babylonian, Akkadian, and Sumerian mythology.
	\item \textbf{Other Names}: Also known as Absu or Abzu.
\end{itemize}

\textbf{{\Large Amurru }}
\begin{itemize}
	\item \textbf{Definition}: God of storms and the sky of the Amorite people who migrated to the Mesopotamian region around 2100 BCE. Associated with Adad but portrayed in a gentler version, often depicted with a gazelle and a shepherd's staff or a shepherd's crook, watching over nomads. Also known as Martu.
	\item \textbf{Civilization}: Associated with the Amorite and Babylonian civilizations.
	\item \textbf{Classification}: Major god in Babylonian and Amorite mythology.
	\item \textbf{Other Names}: Also known as Martu.
\end{itemize}

\textbf{{\Large Anshar }}
\begin{itemize}
	\item \textbf{Definition}: God of the sky of 'An', translated as 'heaven' and 'shar' as 'complete' or 'entire,' signifying the god of the entire sky. Different from Anu, as he represents only the sky and not the sky above the clouds. One of the children of the primordial gods Apsu and Tiamat, and the consort of Kishar. Anshar and Kishar symbolized the sky and the earth, respectively.
	\item \textbf{Civilization}: Associated with Babylonian and Akkadian civilizations.
	\item \textbf{Classification}: Major god in Babylonian and Akkadian mythology.
	\item \textbf{Other Names}: No known alternative names.
\end{itemize}

\textbf{{\Large Antum }}
\begin{itemize}
	\item \textbf{Definition}: Goddess of the earth and one of the early goddesses of fertility.
	\item \textbf{Civilization}: Associated with Babylonian culture.
	\item \textbf{Classification}: Minor goddess in Babylonian mythology.
	\item \textbf{Other Names}: No known alternative names.
\end{itemize}

\textbf{{\Large Anu }}
\begin{itemize}
	\item \textbf{Definition}: God of the sky and Lord of the Heavens in tales written before 2500 BCE. Consort of Antu, and from their union, the Annunaki, the judges of the dead, were born. The Sumerian word "An" translates to "heavens" and he was associated with thunder rolling through the skies. During storms, he was imagined as a mighty bull roaring above the clouds. Over time, Anu became the supreme lord, the power behind all other deities. Only his son Enlil had direct access to him, and people would pray to lesser gods to convey their requests up the chain to Enlil. Anu is the first to hold the Tablets of Destiny before passing them on to Enlil.
	\item \textbf{Civilization}: Associated with Sumerian and Babylonian civilizations.
	\item \textbf{Classification}: Major god in Sumerian and Babylonian mythology.
	\item \textbf{Other Names}: Also known as An in the Sumerian pantheon.
\end{itemize}

\textbf{{\Large Anunnaki }}
\begin{itemize}
	\item \textbf{Definition}: Gods of the "fates" or judges of the dead in Mesopotamian mythology, born from the union of Anu and Antu. In Babylonian mythologies, they were considered earth spirits but still portrayed in the role of judges or "those who see."
	\item \textbf{Civilization}: Associated with Babylonian and Sumerian civilizations.
	\item \textbf{Classification}: Major gods in Babylonian and Sumerian mythology.
	\item \textbf{Other Names}: No known alternative names.
\end{itemize}

\textbf{{\Large Arazu }}
\begin{itemize}
	\item \textbf{Definition}: God of complete construction in Babylonian mythology. Worshiped upon the completion of building projects.
	\item \textbf{Civilization}: Associated with Babylonian culture.
	\item \textbf{Classification}: Minor god in Babylonian mythology.
	\item \textbf{Other Names}: No known alternative names.
\end{itemize}

\textbf{{\Large Aruru }}
\begin{itemize}
	\item \textbf{Definition}: Goddess of nature and a primordial mother goddess who created human beings in conjunction with Enki (sometimes Enlil).
	\item \textbf{Civilization}: Associated with Babylonian culture.
	\item \textbf{Classification}: Minor goddess in Babylonian mythology.
	\item \textbf{Other Names}: No known alternative names.
\end{itemize}

\textbf{{\Large Ashnan }}
\begin{itemize}
	\item \textbf{Definition}: Goddess of grain in Sumerian mythology. Ashnan and her sister, Lahar, were children of Enlil, born to sustain the Annunaki, the judges of the dead. However, it was discovered that the Annunaki could not consume them, and that humans were created to eat the grains, so the efforts of Ashnan and Lahar would not be wasted.
	\item \textbf{Civilization}: Associated with Sumerian civilization.
	\item \textbf{Classification}: Minor goddess in Sumerian mythology.
	\item \textbf{Other Names}: No known alternative names.
\end{itemize}

\textbf{{\Large Assur }}
\begin{itemize}
	\item \textbf{Definition}: God of the supreme deity of the Assyrians, originating as a local deity of the city of Ashur. He was the Assyrian god of the sky and war, known as "The Lord of the Whole Heavens." His name (Anshar) means "whole sky" in Akkadian, and he was often invoked as a powerful ally by the Assyrian kings (whose names often contained elements of his, such as in Assurbanipal). He is often depicted as an archer with a feathered mantle, drawing a bow while riding a serpent or dragon. Much of his mythology and iconography (like Marduk's serpent-dragon or his wife Ninlil) are borrowed from Sumerian or Babylonian works.
	\item \textbf{Civilization}: Associated with Assyrian civilization.
	\item \textbf{Classification}: Major god in Assyrian mythology.
	\item \textbf{Other Names}: Also known as Ashur or Anshar in Akkadian.
\end{itemize}

\textbf{{\Large Baba }}
\begin{itemize}
	\item \textbf{Definition}: The Sumerian goddess of Lagash, a local mother goddess and goddess of fertility known as the "Lady of Animals" and "Lady of Abundance."
	\item \textbf{Civilization}: Associated with Sumerian civilization.
	\item \textbf{Classification}: Minor goddess in Sumerian mythology.
	\item \textbf{Other Names}: Bau or Bawa
\end{itemize}

\textbf{{\Large Babbar }}
\begin{itemize}
	\item \textbf{Definition}: God of the sun, which means "illumination" or "The Illuminated."
	\item \textbf{Civilization}: Associated with Sumerian and Babylonian civilizations.
	\item \textbf{Classification}: Major god in Sumerian and Babylonian mythology.
	\item \textbf{Other Names}: Utu/Shamash
\end{itemize}

\textbf{{\Large Basmu }}
\begin{itemize}
	\item \textbf{Definition}: Great Mesopotamian serpent associated alternatively with goddesses of childbirth and fertility or with Ningishzida, an underworld god. In its association with childbirth, Basmu is sometimes depicted with horns, while as a symbol of Ningishzida, it is intertwined around a staff or portrayed as two copulating snakes.
	\item \textbf{Civilization}: Associated with Mesopotamian culture.
	\item \textbf{Classification}: Minor deity in Mesopotamian mythology.
	\item \textbf{Other Names}: No known alternative names.
\end{itemize}

\textbf{{\Large Bel }}
\begin{itemize}
	\item \textbf{Definition}: God of scholars in Babylonian mythology. Associated with Marduk, sometimes depicted as his brother, Bel was highly intelligent and wise. He was the son of Enki (Ea), the god of wisdom.
	\item \textbf{Civilization}: Associated with Babylonian civilization.
	\item \textbf{Classification}: Major god in Babylonian mythology.
	\item \textbf{Other Names}: No known alternative names.
\end{itemize}

\textbf{{\Large Belit-Tseri}}
\begin{itemize}
	\item \textbf{Definition}: Goddess of the Babylonian scribe of the Underworld, she kneels beside the throne of Ereshkigal and records the names of the dead as they enter the realm of darkness. She was referred to as the "Queen of the Desert." Her consort was Amurru, the Amorite god of the sky and nomads.
	\item \textbf{Civilization}: Associated with Babylonian civilization.
	\item \textbf{Classification}: Minor goddess in Babylonian mythology.
	\item \textbf{Other Names}: No known alternative names.
\end{itemize}

\textbf{{\Large Birdu }}
\begin{itemize}
	\item \textbf{Definition}: God of the Babylonian underworld messenger.
	\item \textbf{Civilization}: Associated with Babylonian culture.
	\item \textbf{Classification}: Minor god in Babylonian mythology.
	\item \textbf{Other Names}: No known alternative names.
\end{itemize}

\textbf{{\Large Cara }}
\begin{itemize}
	\item \textbf{Definition}: God of a Sumerian deity known as the Barber of Inanna. He is one of those whom the underworld demons try to take as a substitute for Inanna in the afterlife, after she returns to the underworld in the poem "The Descent of Inanna." He is spared because Inanna tells the demons that Cara is essential to her.
	\item \textbf{Civilization}: Associated with Sumerian civilization.
	\item \textbf{Classification}: Minor god in Sumerian mythology.
	\item \textbf{Other Names}: No known alternative names.
\end{itemize}

\textbf{{\Large Dagon }}
\begin{itemize}
	\item \textbf{Definition}: God of grain and fertility, particularly popular in the Mesopotamian region of the Middle Euphrates, where he also controlled the weather. His qualities eventually merged with those of Adad.
	\item \textbf{Civilization}: Associated with Babylonian culture.
	\item \textbf{Classification}: Major god in Babylonian mythology.
	\item \textbf{Other Names}: Dagan
\end{itemize}

\textbf{{\Large Damu }}
\begin{itemize}
	\item \textbf{Definition}: God of healing in Sumerian mythology. Damu, the son of Gula, the goddess of healing, was considered the intermediary between his mother and mortal physicians.
	\item \textbf{Civilization}: Associated with Sumerian civilization.
	\item \textbf{Classification}: Minor god in Sumerian mythology.
	\item \textbf{Other Names}: No known alternative names.
\end{itemize}

\textbf{{\Large Damkina }}
\begin{itemize}
	\item \textbf{Definition}: Goddess and consort to the Babylonian god Ea (Enki), and mother of the hero god Marduk.
	\item \textbf{Civilization}: Associated with Babylonian civilization.
	\item \textbf{Classification}: Minor goddess in Babylonian mythology.
	\item \textbf{Other Names}: No known alternative names.
\end{itemize}

\textbf{{\Large Dumuzi }}
\begin{itemize}
	\item \textbf{Definition}: God of fertility and shepherds in Sumerian mythology, who was the husband of the goddess Inanna and the brother of Geshtinanna. He takes Inanna's place in the underworld after she is captured and killed there by Ereshkigal, and Geshtinanna offers to take his place. He remains in the underworld for half the year, and Geshtinanna for the other half, thus explaining the cycle of seasons.
	\item \textbf{Civilization}: Associated with Sumerian civilization.
	\item \textbf{Classification}: Major god in Sumerian mythology.
	\item \textbf{Other Names}: No known alternative names.
\end{itemize}

\textbf{{\Large Ea / Enki}}
\begin{itemize}
	\item \textbf{Definition}: God of wisdom and fresh waters in Babylonian mythology. Known as Enki in Sumerian, he was introduced or developed by the Akkadians. He was the god of magic who defeated his father Apsu and created the earth. Ea/Enki was one of the most important and beloved gods in the Mesopotamian pantheon and prominently appears in the story of the Deluge, where he saves humanity by advising the righteous man Atrahasis to build an ark before the floodwaters arrive, and in the famous Descent of Inanna, he provides the means to rescue the goddess from the underworld. He was the wisest among the gods and the patron of artisans, craftsmen, and exorcists. In the story Inanna and the God of Wisdom, he allows himself to get drunk and donates the meh, the gifts of civilization and the possessions of the gods, to Inanna, knowing that she will disperse them to humanity. He is consistently portrayed as a friend of humans and their advocate among the gods.
	\item \textbf{Civilization}: Associated with Babylonian and Sumerian civilizations.
	\item \textbf{Classification}: Major god in Mesopotamian mythology.
	\item \textbf{Other Names}: Enki (Sumerian).
\end{itemize}

\textbf{{\Large Ellil }}
\begin{itemize}
	\item \textbf{Definition}: The god of wind and storms and the King of the Gods before being replaced by Marduk in Babylonian mythology.
	\item \textbf{Civilization}: Associated with Babylonian and Sumerian civilizations.
	\item \textbf{Classification}: Major god in Babylonian mythology.
	\item \textbf{Other Names}: Enlil (Sumerian).
\end{itemize}

\textbf{{\Large Emesh }}
\begin{itemize}
	\item \textbf{Definition}: God of summer and personification of summer in Sumerian mythology. He created the trees and fertile fields and was the brother of Enten, the god of winter. Emesh was described as a farmer.
	\item \textbf{Civilization}: Associated with Sumerian civilization.
	\item \textbf{Classification}: Minor god in Sumerian mythology.
	\item \textbf{Other Names}: No known alternative names.
\end{itemize}

\textbf{{\Large Enbilulu }}
\begin{itemize}
	\item \textbf{Definition}: God of the Mesopotamian water deity in charge of caring for the Tigris and Euphrates rivers.
	\item \textbf{Civilization}: Associated with Mesopotamian culture.
	\item \textbf{Classification}: Minor god in Mesopotamian mythology.
	\item \textbf{Other Names}: No known alternative names.
\end{itemize}

\textbf{{\Large Enkimdu }}
\begin{itemize}
	\item \textbf{Definition}: God of canals and ditches in Mesopotamian mythology. Like Emesh, he was described as a farmer with a plow and yoke. He was also the god of farmers, fields, and grains.
	\item \textbf{Civilization}: Associated with Mesopotamian culture.
	\item \textbf{Classification}: Minor god in Mesopotamian mythology.
	\item \textbf{Other Names}: No known alternative names.
\end{itemize}

\textbf{{\Large Enkidu }}
\begin{itemize}
	\item \textbf{Definition}: God of forests and wilderness in Sumerian mythology. Created by the gods and sent to earth to teach the proud king Gilgamesh a lesson in humility, Enkidu became Gilgamesh's best friend and brother. His death, following the slaying of the Bull of Heaven, is the impetus for Gilgamesh embarking on his quest for the meaning of life and mortality.
	\item \textbf{Civilization}: Associated with Sumerian civilization.
	\item \textbf{Classification}: Major god in Sumerian mythology.
	\item \textbf{Other Names}: No known alternative names.
\end{itemize}

\textbf{{\Large Enlil }}
\begin{itemize}
	\item \textbf{Definition}: God of air in Sumerian mythology, whose name means "Lord of Air and Wind," but much more powerful than any mere elemental deity. His consort was Ninlil. Enlil, Anu, and Enki formed a triad that ruled over Heaven, Earth, and the Underworld or, alternatively, the heavens, the sky, the atmosphere, and the earth. Enlil was an important weather god, often prayed to and worshipped in hopes of good weather for a bountiful harvest. Holder of the Tablets of Destiny, he was the Lord of the Sumerian pantheon after 2500 BCE and worshipped by the Akkadians c. 2334–c. 2083 BCE. Later, he was absorbed by the god Marduk during the reign of Hammurabi (1792-1750 BCE). Enlil is depicted in various myths as a supreme deity and king of the gods. Although his cult center was in Nippur, he was widely venerated throughout Mesopotamia.
	\item \textbf{Civilization}: Associated with Sumerian civilization.
	\item \textbf{Classification}: Major god in Sumerian mythology.
	\item \textbf{Other Names}: No known alternative names.
\end{itemize}

\textbf{{\Large Enmessara }}
\begin{itemize}
	\item \textbf{Definition}: God of the underworld in Sumerian mythology.
	\item \textbf{Civilization}: Associated with Sumerian civilization.
	\item \textbf{Classification}: Minor god in Sumerian mythology.
	\item \textbf{Other Names}: No known alternative names.
\end{itemize}

\textbf{{\Large Enten }}
\begin{itemize}
	\item \textbf{Definition}: God of winter who cared for the birth and health of animals during the cold and rainy season in Sumerian mythology. His brother was Emesh, the god of summer.
	\item \textbf{Civilization}: Associated with Sumerian civilization.
	\item \textbf{Classification}: Minor god in Sumerian mythology.
	\item \textbf{Other Names}: No known alternative names.
\end{itemize}

\textbf{{\Large Ereshkigal }}
\begin{itemize}
	\item \textbf{Definition}: Goddess of the underworld and Queen of the Dead in Sumerian mythology, whose name means "Lady of the Great Place." Ereshkigal was an important and feared goddess whose consort was the Bull of Heaven until he was killed by Enkidu. She was the older sister of the goddess Inanna, whom she blamed for the death of the Bull of Heaven and whom she killed when Inanna came to visit her in the underworld for his funeral. Due to the cleverness of Enki, she is forced to release Inanna back to the land of the living. She rules over the land of the dead alone (known to the Mesopotamians generally as "The Land Without Return," a dark and gloomy place) until the arrival of the god Nergal who becomes her consort. She was also known as Irkalla. Tales about Ereshkigal, or The Descent of Inanna, have similarities to the later Egyptian myth of Osiris and Isis and the Greek myth of Demeter and Persephone in the motif of the Dying and Reviving god, most famously known from the story of Jesus Christ.
	\item \textbf{Civilization}: Associated with Sumerian civilization.
	\item \textbf{Classification}: Major goddess in Sumerian mythology.
	\item \textbf{Other Names}: Irkalla
\end{itemize}

\textbf{{\Large Erragal }}
\begin{itemize}
	\item \textbf{Definition}: God of the underworld in Sumerian mythology.
	\item \textbf{Civilization}: Associated with Sumerian civilization.
	\item \textbf{Classification}: Minor god in Sumerian mythology.
	\item \textbf{Other Names}: No known alternative names.
\end{itemize}

\textbf{{\Large Erra }}
\begin{itemize}
	\item \textbf{Definition}: God of war, destruction, death, and conflicts in Babylonian mythology, also known as Nergal. Most well-known from the work "The Wrath of Erra," in which he destroys Babylon without reason after tricking Marduk into leaving the city. See IRRA.
	\item \textbf{Civilization}: Associated with Babylonian civilization.
	\item \textbf{Classification}: Major god in Babylonian mythology.
	\item \textbf{Other Names}: Nergal (alternative name in Mesopotamian mythology).
\end{itemize}


2 / 2

\textbf{{\Large Garra }}
\begin{itemize}
	\item \textbf{Definition}: God of fire in Babylonian mythology, especially known for purifying or cleansing fire. Also known as Gerra.
	\item \textbf{Civilization}: Associated with Babylonian civilization.
	\item \textbf{Classification}: Minor god in Babylonian mythology.
	\item \textbf{Other Names}: Gerra (alternative name).
\end{itemize}

\textbf{{\Large Geshtinanna }}
\begin{itemize}
	\item \textbf{Definition}: Goddess of fertility in Sumerian mythology and sister of Dumuzi. Her name means "The Vine of Heaven." She was in charge of fertility of the earth, from spring until the autumn equinox, when she would descend to the underworld to release Dumuzi (who took Inanna's place), and he would then return to Earth to oversee fertility for the next six months of the year.
	\item \textbf{Civilization}: Associated with Sumerian civilization.
	\item \textbf{Classification}: Minor goddess in Sumerian mythology.
	\item \textbf{Other Names}: No known alternative names.
\end{itemize}

\textbf{{\Large Geshtu }}
\begin{itemize}
	\item \textbf{Definition}: The god who offered his blood and intellect for use in the creation of human beings in the Akkadian/Babylonian myth "The Atrahasis."
	\item \textbf{Civilization}: Associated with Akkadian/Babylonian mythology.
	\item \textbf{Classification}: Not considered a deity, but a significant figure in the mythology.
	\item \textbf{Other Names}: Geshtu-e, We-llu.
\end{itemize}

\textbf{{\Large Gibil }}
\begin{itemize}
	\item \textbf{Definition}: God of the Assyrian governor and judge over gods and humans. Known as the ruler of the gods. Associated with judges and had a keen interest in punishing those who had been unjust judges in life. Also the name of a fire god.
	\item \textbf{Civilization}: Associated with Assyrian mythology.
	\item \textbf{Classification}: Minor god in Assyrian mythology.
	\item \textbf{Other Names}: No known alternative names.
\end{itemize}

\textbf{{\Large Gishida }}
\begin{itemize}
	\item \textbf{Definition}: God of the Babylonian Tree of Life and the beginning of spring. He is a dying and reviving god who, along with Tammuz, stands guard at the gates of heaven in the Myth of Adapa. The god Ea tells Adapa to acknowledge the "disappearance of two gods from the earth" by showing respect to Gishida and Tammuz, both leaving the earth for part of the year (thus explaining the change in seasons). Also known as Ningishzida (see NINGISHZIDA).
	\item \textbf{Civilization}: Associated with Babylonian mythology.
	\item \textbf{Classification}: Minor god in Babylonian mythology.
	\item \textbf{Other Names}: Ningishzida (alternative name).
\end{itemize}

\textbf{{\Large Gugalanna }}
\begin{itemize}
	\item \textbf{Definition}: The Sumerian Bull of Heaven and the first consort of the goddess Ereshkigal.
	\item \textbf{Civilization}: Associated with Sumerian mythology.
	\item \textbf{Classification}: Mythological creature in Sumerian mythology.
	\item \textbf{Other Names}: Bull of Heaven (alternative name).
\end{itemize}

\textbf{{\Large Gula }}
\begin{itemize}
	\item \textbf{Definition}: Goddess of healing in Sumerian mythology. Gula was the Sumerian goddess of healing, consort of Ninurta, Pabilsag, and Abu. She was also associated with agriculture and growth. She was the patroness of doctors and healing arts and is often depicted surrounded by stars with her dog by her side. She is associated with the underworld and transformation. Mother of Damu, Ninazu, and Gunurra, all healing deities who are also linked to transformation/transition.
	\item \textbf{Civilization}: Associated with Sumerian civilization.
	\item \textbf{Classification}: Major goddess in Sumerian mythology.
	\item \textbf{Other Names}: Ninkarrak, Ninisina, Bau.
\end{itemize}

\textbf{{\Large Gushkin-Banda}}
\begin{itemize}
	\item \textbf{Definition}: The Babylonian creator of mankind and gods, described as an artisan, often a goldsmith.
	\item \textbf{Civilization}: Associated with Babylonian mythology.
	\item \textbf{Classification}: Not considered a deity, but a significant figure in the mythology.
	\item \textbf{Other Names}: No known alternative names.
\end{itemize}
\newpage
\textbf{{\Large Haia }}
\begin{itemize}
	\item \textbf{Definition}: God of warehouses and goods in Sumerian mythology. He was best known as the father of the grain goddess Ninlil, whose rape by Enlil formed the basis of the famous fertility myth.
	\item \textbf{Civilization}: Associated with Sumerian civilization.
	\item \textbf{Classification}: Minor god in Sumerian mythology.
	\item \textbf{Other Names}: No known alternative names.
\end{itemize}

\textbf{{\Large Igigi }}
\begin{itemize}
	\item \textbf{Definition}: God of the heavens in Babylonian mythology, the region above the clouds. Also the collective name for the gods who dwelled above the clouds.
	\item \textbf{Civilization}: Associated with Babylonian mythology.
	\item \textbf{Classification}: Group of gods in Babylonian mythology.
	\item \textbf{Other Names}: No known alternative names.
\end{itemize}

\textbf{{\Large Imdugud }}
\begin{itemize}
	\item \textbf{Definition}: The Sumerian version of Anzu, Pazusu, and Zu, who had the tendency to turn storms into whirlwinds by flapping its wings. It was especially venerated in the region around the city of Ur.
	\item \textbf{Civilization}: Associated with Sumerian mythology.
	\item \textbf{Classification}: Mythological creature in Sumerian mythology.
	\item \textbf{Other Names}: No known alternative names.
\end{itemize}

\textbf{{\Large Inanna }}
\begin{itemize}
	\item \textbf{Definition}: Goddess of sexuality, passion, fertility, love, prostitutes, and war in Sumerian mythology. Inanna became closely identified with the Babylonian goddess Ishtar, who, over time, assumed many of her attributes. The most popular and beloved of all Sumerian pantheons, Inanna prominently appears in many of the best-known and most frequently copied Sumerian stories, myths, and hymns (including "The Descent of Inanna," "Inanna and the god of Wisdom," "The Courtship of Inanna and Dumuzi," and "The Huluppu Tree") and is listed among the seven primary deities of Sumer early on along with Anu, Enlil, Enki, Ninhursag, Nanna, and Utu. Sargon of Akkad invoked Inanna for protection and victory in battle and for political guidance, and his daughter, Enheduanna, was the High Priestess of Inanna in Uruk and the composer of many hymns and songs to her. She is often depicted riding a lion and is called "The Queen of Heaven." Her elder sister was Ereshkigal. She was the main goddess and patroness of the city of Uruk, to whom she was said to have given the sacred "me" (laws) that were given to her in a drunken state by the god of wisdom, Enki. Many compositions regarding her describe her as highly sexual, unmarried, and capable of "turning men into women" with passion. She was associated with the planet Venus. In the myth of Etana, she is referenced as Innina and, from early on, was regarded as the twin sister of Utu (Shamash), the sun god.
	\item \textbf{Civilization}: Associated with Sumerian civilization.
	\item \textbf{Classification}: Major goddess in Sumerian mythology.
	\item \textbf{Other Names}: Ishtar (alternative name).
\end{itemize}

\textbf{{\Large Irkalla }}
\begin{itemize}
	\item \textbf{Definition}: The Sumerian goddess of the underworld and Queen of the Dead, whose name means "Lady of the Great Place." Ereshkigal was an important and feared goddess whose consort was the Bull of Heaven until he was killed by Enkidu. She was the older sister of the goddess Inanna, whom she blamed for the death of the Bull of Heaven and whom she killed when Inanna came to visit her in the underworld for her funeral. Due to the cleverness of Enki, she is forced to release Inanna back to the land of the living. She rules the land of the dead alone (commonly known by the Mesopotamians as "The Land Without Return," a dark and gloomy place) until the arrival of the god Nergal, who becomes her consort. She was also known as Irkalla. Tales about Ereshkigal, such as "The Descent of Inanna," have similarities to the later Egyptian myth of Osiris and Isis and the Greek myth of Demeter and Persephone in the motif of the Dying and Reviving god, most famously known from the story of Jesus Christ.
	\item \textbf{Civilization}: Associated with Sumerian mythology.
	\item \textbf{Classification}: Major goddess in Sumerian mythology.
	\item \textbf{Other Names}: Ereshkigal, Irkalla (alternative name).
\end{itemize}

\textbf{{\Large Irra }}
\begin{itemize}
	\item \textbf{Definition}: God of plagues, pestilence, death, war, and destruction in Babylonian mythology, associated with Nergal, the god of death. Irra was an intelligent and troublesome demon responsible for all kinds of human misery. In "The Epic of Irra" (also known as "The Wrath of Erra"), he takes over the city of Babylon in the absence of Marduk, leading to its destruction or, alternatively, frees Babylon from its enemies but only after "turning the world upside down" by indiscriminately killing both the righteous and the wicked. The text of "The Epic of Irra" was immensely popular in Babylonia. More copies of this work have been discovered than copies of the most famous "Epic of Gilgamesh."
	\item \textbf{Civilization}: Associated with Babylonian mythology.
	\item \textbf{Classification}: Minor god in Babylonian mythology.
	\item \textbf{Other Names}: Erra (alternative name).
\end{itemize}

\textbf{{\Large Ishara }}
\begin{itemize}
	\item \textbf{Definition}: The Mesopotamian goddess of oaths, known as the "Queen of Judgment." She was also associated with love, war, and divination and sometimes appears as a mother goddess or an underworld deity. According to scholar Jeremy Black, she was "more closely linked to the Semitic tradition than to Sumerian" and was assimilated into Inanna. She was associated with the god Dagan.
	\item \textbf{Civilization}: Associated with Mesopotamian mythology.
	\item \textbf{Classification}: Minor goddess in Mesopotamian mythology.
	\item \textbf{Other Names}: No known alternative names.
\end{itemize}

\textbf{{\Large Ishum }}
\begin{itemize}
	\item \textbf{Definition}: The Babylonian god of fire.
	\item \textbf{Civilization}: Associated with Babylonian mythology.
	\item \textbf{Classification}: God in Babylonian mythology.
	\item \textbf{Other Names}: No known alternative names.
\end{itemize}

\textbf{{\Large Kabta }}
\begin{itemize}
	\item \textbf{Definition}: The Sumerian god of picks, construction, and bricks, the brother of Mushdamma (god of foundations and buildings), and one of the many children of Ninhursag.
	\item \textbf{Civilization}: Associated with Sumerian mythology.
	\item \textbf{Classification}: God in Sumerian mythology.
	\item \textbf{Other Names}: No known alternative names.
\end{itemize}

\textbf{{\Large Kishar }}
\begin{itemize}
	\item \textbf{Definition}: The Babylonian goddess of the earth and fertility. As her name, 'Ki' and 'Shar' meaning 'the entire earth,' she is considered a mother goddess responsible for growth both below and above the ground. She is the mother of Anu in some myths and associated with Anshar (sky).
	\item \textbf{Civilization}: Associated with Babylonian mythology.
	\item \textbf{Classification}: Goddess in Babylonian mythology.
	\item \textbf{Other Names}: No known alternative names.
\end{itemize}

\textbf{{\Large Kittu }}
\begin{itemize}
	\item \textbf{Definition}: The Sumerian god of justice, the brother of Misharu. In some sources, Misharu is the god of law and Kittu is the god of justice that proceeds from the law, while in others, they appear to be equal, both being gods of law and justice. See \textbf{Misharu}.
	\item \textbf{Civilization}: Associated with Sumerian mythology.
	\item \textbf{Classification}: God in Sumerian mythology.
	\item \textbf{Other Names}: No known alternative names.
\end{itemize}

\textbf{{\Large Kulitta }}
\begin{itemize}
	\item \textbf{Definition}: The Babylonian goddess of music who served Ishtar with beautiful songs for Tammuz.
	\item \textbf{Civilization}: Associated with Babylonian mythology.
	\item \textbf{Classification}: Goddess in Babylonian mythology.
	\item \textbf{Other Names}: No known alternative names.
\end{itemize}

\textbf{{\Large Kulla }}
\begin{itemize}
	\item \textbf{Definition}: The Babylonian god who restored temples, also known as the god of the brick, who, like Kabta and Mushdamma, was invoked in the laying of the foundation of buildings and praised or sent away upon completion. Tablets from the Akkadian period provide incantations for blessings from Kulla at the beginning of a project, as well as for banishing the god from the site once the building was done, as it was believed he might linger when there were others who needed him, and furthermore, his presence could signify more construction was needed.
	\item \textbf{Civilization}: Associated with Babylonian mythology.
	\item \textbf{Classification}: God in Babylonian mythology.
	\item \textbf{Other Names}: No known alternative names.
\end{itemize}

\textbf{{\Large Kulullu }}
\begin{itemize}
	\item \textbf{Definition}: The collective name for water spirits, elemental spirits of streams and lakes, in Assyrian mythology.
	\item \textbf{Civilization}: Associated with Assyrian mythology.
	\item \textbf{Classification}: Mythological beings in Assyrian mythology.
	\item \textbf{Other Names}: No known alternative names.
\end{itemize}

\textbf{{\Large Kusag }}
\begin{itemize}
	\item \textbf{Definition}: The Babylonian god of priesthood, the patron god of priests. Kusag is the High Priest among the gods and officiates in their rituals.
	\item \textbf{Civilization}: Associated with Babylonian mythology.
	\item \textbf{Classification}: God in Babylonian mythology.
	\item \textbf{Other Names}: No known alternative names.
\end{itemize}

\textbf{{\Large Kur }}
\begin{itemize}
	\item \textbf{Definition}: The Sumerian word for "mountain" that referred to both the high dwelling place of the gods and the vast underworld below the Earth's surface, depending on the context.
	\item \textbf{Civilization}: Associated with Sumerian mythology.
	\item \textbf{Classification}: Concept in Sumerian mythology.
	\item \textbf{Other Names}: No known alternative names.
\end{itemize}

\textbf{{\Large Lahar }}
\begin{itemize}
	\item \textbf{Definition}: The Sumerian goddess of cattle. She and her sister, Anshar, were originally created to feed and adorn the Anunnaki gods. See \textbf{Anshar}.
	\item \textbf{Civilization}: Associated with Sumerian mythology.
	\item \textbf{Classification}: Goddess in Sumerian mythology.
	\item \textbf{Other Names}: No known alternative names.
\end{itemize}

\textbf{{\Large Lahmu and Lahamu}}
\begin{itemize}
	\item \textbf{Definition}: The first two gods of Babylon, the first-born of Absu and Tiamat, from whom all other gods were born.
	\item \textbf{Civilization}: Associated with Babylonian mythology.
	\item \textbf{Classification}: Gods in Babylonian mythology.
	\item \textbf{Other Names}: No known alternative names.
\end{itemize}

\textbf{{\Large Lama }}
\begin{itemize}
	\item \textbf{Definition}: The Sumerian goddess of protection. She was known to the Akkadians as Lamassu. As Lama, she was depicted as a woman with a long mantle, while as Lamassu, she appeared as a winged bull or lion with a woman's face or head and protected temples and palaces from the forces of chaos and their accompanying evils. Lama frequently appears on cylinder seals and was widely sought for intercession with the gods. Her name means "protective spirit."
	\item \textbf{Civilization}: Associated with Sumerian mythology.
	\item \textbf{Classification}: Goddess in Sumerian mythology.
	\item \textbf{Other Names}: Lamassu (alternative name).
\end{itemize}

\textbf{{\Large Mammetum }}
\begin{itemize}
	\item \textbf{Definition}: The Akkadian goddess of fate and destiny. According to some myths, she dwelled in the underworld and simply decreed the fate of humans on a whim, but the fate she decreed would come to pass.
	\item \textbf{Civilization}: Associated with Akkadian mythology.
	\item \textbf{Classification}: Goddess in Akkadian mythology.
	\item \textbf{Other Names}: Mamitu.
\end{itemize}

\textbf{{\Large Marduk }}
\begin{itemize}
	\item \textbf{Definition}: The Babylonian king of the gods, the hero-god who defeated Tiamat and the forces of chaos and brought order to the universe that the gods and humans work together to maintain. He is the god of healing, justice, compassion, regeneration, magic, and righteousness. He was known as the peacemaker among the gods and was referred to, in this regard, as the "Shepherd of the Gods." In the Epic of Irra, Marduk leaves the city of Babylon in the hands of Nergal (Irra, Erra), who destroys it in fury. Marduk was one of the most popular and enduring gods of Mesopotamia and was adopted by the Assyrians as the son of their supreme god, Ashur.
	\item \textbf{Civilization}: Associated with Babylonian mythology.
	\item \textbf{Classification}: God in Babylonian mythology.
	\item \textbf{Other Names}: No known alternative names.
\end{itemize}

\textbf{{\Large Misharu }}
\begin{itemize}
	\item \textbf{Definition}: The Sumerian god of law and justice, the brother of Kittu.
	\item \textbf{Civilization}: Associated with Sumerian mythology.
	\item \textbf{Classification}: God in Sumerian mythology.
	\item \textbf{Other Names}: No known alternative names.
\end{itemize}

\textbf{{\Large Mummu }}
\begin{itemize}
	\item \textbf{Definition}: The Babylonian god of craftsmen. The god Ea is known as Ea Mummu in his role as the creator of human beings, and the word "mummu" is understood as "genius" as in "knowing how to create."
	\item \textbf{Civilization}: Associated with Babylonian mythology.
	\item \textbf{Classification}: God in Babylonian mythology.
	\item \textbf{Other Names}: Ea, Ea Mummu.
\end{itemize}

\textbf{{\Large Mushdamma }}
\begin{itemize}
	\item \textbf{Definition}: The Sumerian god of foundations and buildings, the brother of Kabta (god of picks, construction, and bricks), and one of the sons of Ninhursag.
	\item \textbf{Civilization}: Associated with Sumerian mythology.
	\item \textbf{Classification}: God in Sumerian mythology.
	\item \textbf{Other Names}: No known alternative names.
\end{itemize}

\textbf{{\Large Mylitta }}
\begin{itemize}
	\item \textbf{Definition}: The Assyrian goddess of fertility and childbirth.
	\item \textbf{Civilization}: Associated with Assyrian mythology.
	\item \textbf{Classification}: Goddess in Assyrian mythology.
	\item \textbf{Other Names}: No known alternative names.
\end{itemize}


\textbf{{\Large Nabu }}
\begin{itemize}
	\item \textbf{Definition}: The Babylonian god of writing and wisdom, son of Marduk and grandson of the god of wisdom, Enki (Ea). His name means "The Proclaimer" in reference to his prophetic abilities and gift of writing. He was the patron god of scribes and safeguarded the Tablets of Destiny, which legitimized the ruler of the universe (alternately given as Anu, Enlil, or Marduk, and later Assur). He is depicted holding a stylus and standing or riding alongside a Mushhushshu dragon. Nabu was one of the most important gods in Mesopotamia, and without him, the great Babylonian Akitu festival, held to honor the gods and give thanks for the harvest, could not be celebrated. He was venerated for thousands of years and among the few that survived the fall of the Assyrian Empire in 612 BCE when the statues and temples of many gods were plundered by invading forces. He is often compared to Thoth of the Egyptians, Apollo by the Greeks, and Mercury by the Romans.
	\item \textbf{Civilization}: Associated with Babylonian mythology.
	\item \textbf{Classification}: God in Babylonian mythology.
	\item \textbf{Other Names}: No known alternative names.
\end{itemize}

\textbf{{\Large Nammu }}
\begin{itemize}
	\item \textbf{Definition}: The Sumerian goddess of the primeval sea and the initial abyss, a mother goddess. There is also a Babylonian goddess of the same name who is associated with fresh water but is a minor deity.
	\item \textbf{Civilization}: Associated with Sumerian and Babylonian mythology.
	\item \textbf{Classification}: Goddess in Sumerian and Babylonian mythology.
	\item \textbf{Other Names}: No known alternative names.
\end{itemize}

\textbf{{\Large Namtar }}
\begin{itemize}
	\item \textbf{Definition}: The Sumerian god-demon of fate regarding death, also known as the Herald of Death, associated with Ereshkigal. He carries messages from the underworld to the upper realms of the gods. He was famously insulted by Nergal at the feast given by the gods, where he represented Ereshkigal because she could not attend. This insult results in Nergal eventually falling in love with Ereshkigal and living with her in the underworld.
	\item \textbf{Civilization}: Associated with Sumerian mythology.
	\item \textbf{Classification}: God in Sumerian mythology.
	\item \textbf{Other Names}: No known alternative names.
\end{itemize}

\textbf{{\Large Nana }}
\begin{itemize}
	\item \textbf{Definition}: A virgin mother goddess whose attributes were assumed by Inanna.
	\item \textbf{Civilization}: Associated with Sumerian mythology.
	\item \textbf{Classification}: Goddess in Sumerian mythology.
	\item \textbf{Other Names}: No known alternative names.
\end{itemize}

\textbf{{\Large Nanaja }}
\begin{itemize}
	\item \textbf{Definition}: A Sumerian goddess of sex and war whose attributes were assumed by Inanna.
	\item \textbf{Civilization}: Associated with Sumerian mythology.
	\item \textbf{Classification}: Goddess in Sumerian mythology.
	\item \textbf{Other Names}: No known alternative names.
\end{itemize}

\textbf{{\Large Nanna }}
\begin{itemize}
	\item \textbf{Definition}: Also known as Nanna-Suen, Nannar, Sin, the Sumerian god of the full moon and wisdom, father of Inanna in some stories, and father, with Ningal, of Utu/Shamash, the god of the sun.Nanna is the son of Enlil and Ninlil and is considered an important deity in the act of creation. His symbol is the crescent moon, and he was associated with the power of the bull and the lion-dragon. Nanna is one of the oldest gods in the Mesopotamian pantheon and is first mentioned in early Sumerian writing around 3500 BCE.
	\item \textbf{Civilization}: Associated with Sumerian mythology.
	\item \textbf{Classification}: God in Sumerian mythology.
	\item \textbf{Other Names}: Nanna-Suen, Nannar, Sin.
\end{itemize}

\textbf{{\Large Nanibgal }}
\begin{itemize}
	\item \textbf{Definition}: A minor goddess, consort of Ennugi, the god of dikes and canals. She is often associated with the goddess Nisaba (Nidaba) in her capacity as a keeper of accounts/records.
	\item \textbf{Civilization}: Associated with Sumerian mythology.
	\item \textbf{Classification}: Goddess in Sumerian mythology.
	\item \textbf{Other Names}: No known alternative names.
\end{itemize}

\textbf{{\Large Nanshe }}
\begin{itemize}
	\item \textbf{Definition}: The Sumerian goddess of social justice who cared for orphans and widows. She also oversaw justice, freshwater, birds and fish, fertility, and favored prophets by giving them the ability to interpret dreams. She was also known as the Lady of Warehouses and, in that capacity, oversaw that weights and measures were correct. Her consort was Haia, the god of storehouses. She was adept at dream interpretation, and in a famous myth, the pious king Gudea consulted her about a dream concerning the appropriate time to build a temple.
	\item \textbf{Civilization}: Associated with Sumerian mythology.
	\item \textbf{Classification}: Goddess in Sumerian mythology.
	\item \textbf{Other Names}: Nanse.
\end{itemize}

\textbf{{\Large Nedu }}
\begin{itemize}
	\item \textbf{Definition}: The Babylonian guardian of the gates of the underworld.
	\item \textbf{Civilization}: Associated with Babylonian mythology.
	\item \textbf{Classification}: God in Babylonian mythology.
	\item \textbf{Other Names}: No known alternative names.
\end{itemize}

\textbf{{\Large Nergal }}
\begin{itemize}
	\item \textbf{Definition}: The Sumerian god of war, pestilence, destruction, death, and the underworld, co-regent with Ereshkigal, but originally associated with Shamash, the sun god and a solar deity. His cult center was in Kutha, where he was known as Meslamtaea, an agricultural god associated with the heat of the sun in its negative aspects. The intensity of the summer sun (or the midday sun) was believed to be caused by the fury of Meslamtaea and shifted from being a regional god to a universal god associated with the negative aspects of life. Nergal is best known for insulting Namtar, the representative of Ereshkigal at the gods' feast, and having to make amends with her, resulting in his love affair and eventual move to the underworld to live with her. In some myths, he is credited with the creation of human beings, and in incantations, he is invoked for protection because of his great strength. As Erra, he is famous from the work The Wrath of Erra in which he destroys Babylon without reason.
	\item \textbf{Civilization}: Associated with Sumerian and Babylonian mythology.
	\item \textbf{Classification}: God in Sumerian and Babylonian mythology.
	\item \textbf{Other Names}: Erra, Irra, Meslamtaea.
\end{itemize}

\textbf{{\Large Neti }}
\begin{itemize}
	\item \textbf{Definition}: The Sumerian guardian of the gates of the underworld and a scribe. Neti prominently appears in "The Descent of Inanna."
	\item \textbf{Civilization}: Associated with Sumerian mythology.
	\item \textbf{Classification}: God in Sumerian mythology.
	\item \textbf{Other Names}: No known alternative names.
\end{itemize}

\textbf{{\Large Nidaba }}
\begin{itemize}
	\item \textbf{Definition}: The Sumerian goddess of writing and astrology.
	\item \textbf{Civilization}: Associated with Sumerian mythology.
	\item \textbf{Classification}: Goddess in Sumerian mythology.
	\item \textbf{Other Names}: No known alternative names.
\end{itemize}

\textbf{{\Large Nin-Agal}}
\begin{itemize}
	\item \textbf{Definition}: The Babylonian god of forging and patron deity of blacksmiths.
	\item \textbf{Civilization}: Associated with Babylonian mythology.
	\item \textbf{Classification}: God in Babylonian mythology.
	\item \textbf{Other Names}: No known alternative names.
\end{itemize}

\textbf{{\Large Ninazu }}
\begin{itemize}
	\item \textbf{Definition}: The Babylonian healing god, son of Gula, associated with serpents (symbols of transformation) and the underworld (transition). He possessed a staff of intertwined serpents that was borrowed by the Egyptians and Greeks and is recognized today as the caduceus, the symbol of Hippocrates, the father of medicine.
	\item \textbf{Civilization}: Associated with Babylonian mythology.
	\item \textbf{Classification}: God in Babylonian mythology.
	\item \textbf{Other Names}: No known alternative names.
\end{itemize}

\textbf{{\Large Ningal }}
\begin{itemize}
	\item \textbf{Definition}: A Sumerian goddess of fertility associated with the sun, mother of Utu/Shamash, the sun itself, and wife/consort of Nanna, the moon god. Her name means "Great Lady."
	\item \textbf{Civilization}: Associated with Sumerian mythology.
	\item \textbf{Classification}: Goddess in Sumerian mythology.
	\item \textbf{Other Names}: No known alternative names.
\end{itemize}

\textbf{{\Large Ningishzida }}
\begin{itemize}
	\item \textbf{Definition}: A Sumerian god of the underworld, alternately the son of Ereshkigal and Gugullana or of Anu, the sky god. His symbol was the intertwined Basmu serpent around a staff, much like the later Caduceus of Hermes. He was known as the "Lord of the Good Tree" and was associated with protection and fertility. Also known as Geshida, and under that name, he appears alongside Tammuz as a dying and reviving divine figure who watches over the gates of the gods in the Myth of Adapa. SEE GISHIDA.
	\item \textbf{Civilization}: Associated with Sumerian mythology.
	\item \textbf{Classification}: God in Sumerian mythology.
	\item \textbf{Other Names}: Geshida.
\end{itemize}

\textbf{{\Large Ningizzida }}
\begin{itemize}
	\item \textbf{Definition}: The Babylonian guardian of the gates of heaven who watches over the eastern gate, the most prominent as it is the gate of the morning.
	\item \textbf{Civilization}: Associated with Babylonian mythology.
	\item \textbf{Classification}: God in Babylonian mythology.
	\item \textbf{Other Names}: No known alternative names.
\end{itemize}

\textbf{{\Large Ninhursag }}
\begin{itemize}
	\item \textbf{Definition}: The Sumerian Mother goddess, goddess of fertility, nature, and life on Earth. Her name translates as "Lady of the Mountain's Side," and she was known as the "Mother of the Gods." She was originally a highly popular goddess whose attributes were later assumed by other deities. She is also known as Belet-Ili, Damgalnuna, Ki, Nintu, Nintur, Aruru, Ninmah, Mami, and Mama. Ninhursag appears in many of the most popular myths of Mesopotamia, where she is always associated with life, fertility, growth, and transformation. Her primary duty was to care for women and children, especially pregnant women and young children. She cared for a child from conception, through pregnancy, and after birth, providing nourishment for the child. As with all female deities in Mesopotamia,
	\item \textbf{Civilization}: Associated with Sumerian mythology.
	\item \textbf{Classification}: Goddess in Sumerian mythology.
	\item \textbf{Other Names}: Belet-Ili, Damgalnuna, Ki, Nintu, Nintur, Aruru, Ninmah, Mami, Mama.
\end{itemize}

\textbf{{\Large Nin-Ildu}}
\begin{itemize}
	\item \textbf{Definition}: The Babylonian god of carpentry and patron deity of carpenters.
	\item \textbf{Civilization}: Associated with Babylonian mythology.
	\item \textbf{Classification}: God in Babylonian mythology.
	\item \textbf{Other Names}: No known alternative names.
\end{itemize}

\textbf{{\Large Ninkasi }}
\begin{itemize}
	\item \textbf{Definition}: The Sumerian goddess of alcohol, beer, and brewing, patron deity of brewers. She was said to prepare a fresh batch of beer every day from the finest ingredients. One of her works is the "Hymn to Ninkasi," an ancient Sumerian poem that contains a recipe for brewing beer.
	\item \textbf{Civilization}: Associated with Sumerian mythology.
	\item \textbf{Classification}: Goddess in Sumerian mythology.
	\item \textbf{Other Names}: No known alternative names.
\end{itemize}

\textbf{{\Large Ninurta }}
\begin{itemize}
	\item \textbf{Definition}: The Sumerian god of war, agriculture, and the plow. He was a hero god associated with victory and the vanquisher of chaos and monsters. Ninurta was often depicted as a young man holding a bow and arrow, and he was also associated with the thunderstorm.
	\item \textbf{Civilization}: Associated with Sumerian mythology.
	\item \textbf{Classification}: God in Sumerian mythology.
	\item \textbf{Other Names}: No known alternative names.
\end{itemize}

\textbf{{\Large Nisaba }}
\begin{itemize}
	\item \textbf{Definition}: The Sumerian goddess of writing, learning, and the harvest. She was the patron deity of scribes and scholars and was associated with wisdom and knowledge. Nisaba was often depicted with a stylus and clay tablet, symbolizing her role in recording and preserving information.
	\item \textbf{Civilization}: Associated with Sumerian mythology.
	\item \textbf{Classification}: Goddess in Sumerian mythology.
	\item \textbf{Other Names}: No known alternative names.
\end{itemize}

\textbf{{\Large Nusku }}
\begin{itemize}
	\item \textbf{Definition}: The Babylonian god of light and fire, associated with lamps and torches. Nusku was considered a messenger between the gods and humans and was invoked in rituals to bring light and dispel darkness. He was also associated with the domestic hearth and was believed to protect homes from evil spirits.
	\item \textbf{Civilization}: Associated with Babylonian mythology.
	\item \textbf{Classification}: God in Babylonian mythology.
	\item \textbf{Other Names}: No known alternative names.
\end{itemize}

\textbf{{\Large Nutaggal }}
\begin{itemize}
	\item \textbf{Definition}: The Babylonian goddess of the netherworld, associated with the realm of the dead and the afterlife. Nutaggal was believed to be the wife of Nergal, the god of the underworld, and together they ruled over the souls of the deceased.
	\item \textbf{Civilization}: Associated with Babylonian mythology.
	\item \textbf{Classification}: Goddess in Babylonian mythology.
	\item \textbf{Other Names}: No known alternative names.
\end{itemize}

\textbf{{\Large Nutum }}
\begin{itemize}
	\item \textbf{Definition}: The Babylonian god of irrigation and canals. Nutum was associated with the proper management of water resources and the distribution of water for agricultural purposes. He ensured that the canals and irrigation systems functioned properly to support farming and fertility.
	\item \textbf{Civilization}: Associated with Babylonian mythology.
	\item \textbf{Classification}: God in Babylonian mythology.
	\item \textbf{Other Names}: No known alternative names.
\end{itemize}

\textbf{{\Large Quingu }}
\begin{itemize}
	\item \textbf{Definition}: Consort of Tiamat and champion in her war against the younger gods. Stole the Tablets of Destiny.
	\item \textbf{Civilization}: Babylonian and Sumerian.
	\item \textbf{Classification}: Major god in the Babylonian and Sumerian pantheons.
	\item \textbf{Other Names}: Kingu (alternative spelling).
\end{itemize}

\textbf{{\Large Ramman }}
\begin{itemize}
	\item \textbf{Definition}: The god of Storms and thunder.
	\item \textbf{Civilization}: Acadian and Sumerian.
	\item \textbf{Classification}: Major god in the Acadian and Sumerian pantheons.
	\item \textbf{Other Names}: Rimmon.
\end{itemize}

\textbf{{\Large Sakkan }}
\begin{itemize}
	\item \textbf{Definition}: The god of Livestock and animal fertility.
	\item \textbf{Civilization}: Sumerian.
	\item \textbf{Classification}: Minor god in the Sumerian pantheon.
	\item \textbf{Other Names}: Sumuqan (in Akkadian).
\end{itemize}

\textbf{{\Large Sebitti }}
\begin{itemize}
	\item \textbf{Definition}: Minor warrior gods associated with Nergal and the Pleiades.
	\item \textbf{Civilization}: Babylonian.
	\item \textbf{Classification}: Minor gods in the Babylonian pantheon.
	\item \textbf{Other Names}: None mentioned.
\end{itemize}

\textbf{{\Large Shamash }}
\begin{itemize}
	\item \textbf{Definition}: The god of Sun and justice.
	\item \textbf{Civilization}: Acadian and Sumerian.
	\item \textbf{Classification}: Major god in the Acadian and Sumerian pantheons.
	\item \textbf{Other Names}: Samas, Babbar, Utu.
\end{itemize}

\textbf{{\Large Shara }}
\begin{itemize}
	\item \textbf{Definition}: Minor war deity.
	\item \textbf{Civilization}: Babylonian.
	\item \textbf{Classification}: Minor god in the Babylonian pantheon.
	\item \textbf{Other Names}: None mentioned.
\end{itemize}

\textbf{{\Large Sherida }}
\begin{itemize}
	\item \textbf{Definition}: Ancient Sumerian mother goddess of light and life.
	\item \textbf{Civilization}: Sumerian.
	\item \textbf{Classification}: Major god in the Sumerian pantheon.
	\item \textbf{Other Names}: Serida (alternative spelling), Aja (by Akkadians and Babylonians).
\end{itemize}

\textbf{{\Large Shulpae }}
\begin{itemize}
	\item \textbf{Definition}: The god of Festivities and good times.
	\item \textbf{Civilization}: Mesopotamian.
	\item \textbf{Classification}: Minor god.
	\item \textbf{Other Names}: None mentioned.
\end{itemize}

\textbf{{\Large Shutu }}
\begin{itemize}
	\item \textbf{Definition}: Sumerian personification of the south wind and disease.
	\item \textbf{Civilization}: Sumerian.
	\item \textbf{Classification}: Minor god.
	\item \textbf{Other Names}: None mentioned.
\end{itemize}
\newpage
\textbf{{\Large Silício }}
\begin{itemize}
	\item \textbf{Definition}: Babylonian goddess of horses.
	\item \textbf{Civilization}: Babylonian.
	\item \textbf{Classification}: Minor goddess.
	\item \textbf{Other Names}: "The Divine Mare."
\end{itemize}

\textbf{{\Large Pecado }}
\begin{itemize}
	\item \textbf{Definition}: Moon and associated with the fertility of women and livestock.
	\item \textbf{Civilization}: Babylonian (known as Nanna) and Sumerian.
	\item \textbf{Classification}: Major god in the Babylonian and Sumerian pantheons.
	\item \textbf{Other Names}: Suen, Nanna-Suen, Nanar.
\end{itemize}

\textbf{{\Large Sumuqan }}
\begin{itemize}
	\item \textbf{Definition}: Acadian and Babylonian version of Sakkan.
	\item \textbf{Civilization}: Acadian and Babylonian.
	\item \textbf{Classification}: Minor god.
	\item \textbf{Other Names}: None mentioned.
\end{itemize}

\textbf{{\Large Sumugan }}
\begin{itemize}
	\item \textbf{Definition}: Sumerian god of open plains and caretaker of grazing animals.
	\item \textbf{Civilization}: Sumerian.
	\item \textbf{Classification}: Minor god.
	\item \textbf{Other Names}: Shumugan.
\end{itemize}

\textbf{{\Large Tammuz }}
\begin{itemize}
	\item \textbf{Definition}: The god of Vegetation, dying, and reviving figure.
	\item \textbf{Civilization}: Babylonian (version of the Sumerian god Dumuzi).
	\item \textbf{Classification}: Minor god.
	\item \textbf{Other Names}: Dumuzi.
\end{itemize}

\textbf{{\Large Tiamat }}
\begin{itemize}
	\item \textbf{Definition}: The goddess of Primal mother goddess, mother of the gods, depicted as a dragon.
	\item \textbf{Civilization}: Mesopotamian.
	\item \textbf{Classification}: Major goddess.
	\item \textbf{Other Names}: None mentioned.
\end{itemize}

\textbf{{\Large Umunmutamkag }}
\begin{itemize}
	\item \textbf{Definition}: The god of Offerings who acted as an intermediary between humans and the gods.
	\item \textbf{Other Names}: None mentioned.
\end{itemize}

\textbf{{\Large Uttu }}
\begin{itemize}
	\item \textbf{Definition}: The goddess of Weaving, textiles, clothing.
	\item \textbf{Civilization}: Sumerian.
	\item \textbf{Classification}: Minor goddess.
	\item \textbf{Other Names}: None mentioned.
\end{itemize}

\textbf{{\Large Utu }}
\begin{itemize}
	\item \textbf{Definition}: The god of Sun, justice.
	\item \textbf{Civilization}: Sumerian.
	\item \textbf{Classification}: Major god.
	\item \textbf{Other Names}: Shamash, Samas, Babbar.
\end{itemize}

\textbf{{\Large We-Llu}}
\begin{itemize}
	\item \textbf{Definition}: The god of Sacrificing himself to create humanity.
	\item \textbf{Civilization}: Sumerian.
	\item \textbf{Classification}: Minor god.
	\item \textbf{Other Names}: Geshtu.
\end{itemize}

\textbf{{\Large Zababa }}
\begin{itemize}
	\item \textbf{Definition}: The god of War.
	\item \textbf{Civilization}: Akkadian.
	\item \textbf{Classification}: Minor god.
	\item \textbf{Other Names}: None mentioned.
\end{itemize}

\textbf{{\Large Zakar }}
\begin{itemize}
	\item \textbf{Definition}: The god of Dreams and dream messages.
	\item \textbf{Civilization}: Babylonian.
	\item \textbf{Classification}: Minor god.
	\item \textbf{Other Names}: Zaqar (Sumerian).
\end{itemize}

\textbf{{\Large Zarpanit }}
\begin{itemize}
	\item \textbf{Definition}: The goddess of Fertility.
	\item \textbf{Civilization}: Sumerian and Babylonian.
	\item \textbf{Classification}: Minor goddess.
	\item \textbf{Other Names}: Beltia.
\end{itemize}

\textbf{{\Large Zaltu }}
\begin{itemize}
	\item \textbf{Definition}: The goddess of Strife.
	\item \textbf{Civilization}: Babylonian.
	\item \textbf{Classification}: Minor goddess.
	\item \textbf{Other Names}: None mentioned.
\end{itemize}

\section*{The Sheets}
All of the vampires here are new at the very beginning of the First City but very old when the deluge happens. It is conceivable that for the second generation all the discplines that are inherited by their progeny are in level 10 but they know at least 5 more at the same level and the remaining between 5 and 8.  

\newpage
\subsection*{Enheduanna}
The suggested discpline levels are:

\begin{description}
	\item[Animalism] 10
	\item[Auspex] 10
	\item[Chimerstry] 10
	\item[Daimoinon] 10
	\item[Dominate] 10
	\item[Fortitude] 10
	\item[Necromancy] 10
	\item[Obfuscation] 10
	\item[Presence] 10
	\item[Protean] 10
	\item[Temporis] 10
	\item[Thanathosis] 10
	\item[Valeren] 10
\end{description}

\newpage
\subsection*{Enoch}
The suggested discpline levels are:

\begin{description}
	\item[Auspex] 10 
	\item[Celerity] 10
	\item[Dementation] 10
	\item[Dominate] 10
	\item[Fortitude] 10
	\item[Obfuscation] 10
	\item[Presence] 10
	\item[Serpentis] 10
	\item[Valeren] 10
\end{description}

\newpage
\subsection*{Irad}

\begin{figure}[H]
	\centering
	\includegraphics[scale=0.69]{art/csheets/2ndgen/bkg-trans-irad}
	\label{fig:irad-sheet}
\end{figure}

\newpage
\subsection*{Zillah}
The suggested discpline levels are:

\begin{description}
	\item[Animalism] 10
	\item[Auspex] 10
	\item[Celerity] 10
	\item[Chimerstry] 10
	\item[Dementation] 10
	\item[Dominate] 10
	\item[Fortitude] 10
	\item[Obfuscation] 10
	\item[Potence] 10
	\item[Presence] 10
	\item[Quietus] 10
	\item[Serpentis] 10
	\item[Spiritus] 10
	\item[Vicissitude] 10 
\end{description}

Remembering that second generation after the curse of Cain still holds far more power than the third generation.  If you take into consideration that third generation would have Plot Device as a level 10 in a specific Discipline consider that second generation could undo any Plot Device they wish, when they wish, no matter where they are and no matter how many plot devices are in place.