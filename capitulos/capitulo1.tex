\chapter*{The Events and The History}
\chapterCiting{History is only the register of crimes and misfortunes}{Voltaire}
There is some misinformation here, so allow me to make a few corrections and provide a revised version of your text:

When I was born, my parents were already cast out of the Garden. I have no idea what it looked like or how things were there. Naturally, I asked about it, but all I received was a raw description, similar to what we can find in scriptures or apocryphal books. My parents simply told me that the Garden of Eden was a harmonious entity, with their spirits tightly bound to the Divine. It existed as a realm on Earth. I am uncertain if it's possible for us to return or if there is anything mankind can do to regain access. Therefore, I will refrain from making false claims.

My name is somewhat unconventional. You see, my mother named me \textbf{Adiaphotos}, and my brother \textbf{Amilabes}. However, for some reason, \textbf{Cain} and \textbf{Abel} became the names we are known by today. I must admit, these names are easier to remember and spell. Nevertheless, I now understand that our destinies were already determined by our original names. Adiaphotos has been interpreted by certain scholars as \textit{one devoid of light}, while Amilabes is said to mean \textit{he who dons the garment of light}. Thank you, Mom and Dad.

I remember that I was my father's firstborn, and I had a sister named \textbf{Luluwa}, though we were not twins. I experienced mixed emotions towards her, as everything was new, and I started feeling somewhat neglected while she was still in our mother's womb. Troubling thoughts crossed my young mind.

We lived in a cave, the very cave we were born in, which was spacious enough for me, my sister, Dad, and Mom. Within it, there was an altar where our parents would pray to God. We often journeyed to the Cave of Treasures and a nearby river, the location of which remains unknown to me. During these trips, Mom and Dad would pray fervently. Eventually, we would return to our birth cave, where God blessed my mother with twins, Abel and his sister \textbf{Aklemia}, who were three years younger than us.

As we grew older, I observed that my younger brother possessed a gentle heart, always obedient to our parents. He would stay close to them, eager to learn, constantly moving back and forth between them.

Abel truly cherished our bond, and I reciprocated those feelings. Whenever he returned to our birth cave with Mom and Dad, he would request a moment alone with me. We would wander around the entrance of our cave, and he would share a report of the day's events, as recorded in some old texts. Some parts of the content I will now share are derived from a diary that our parents inscribed on tablets and ancient fiber tissue, while others are my personal recollections of those experiences.

\section*{The tale of Abel and Satan}

As Abel entered the Cave of Treasures, his gaze fell upon the gleaming golden rods, the fragrant incense, and the precious myrrh. Curiosity ignited within him, prompting him to inquire about these captivating artifacts from our parents.

\begin{dialogue}
	\speak{Abel}"How did you come by these?"
\end{dialogue}


Dad told him all that had befallen them. And Abel felt deeply about that. Dad told him of the works of God, and of the garden; and after that, he remained behind father the whole of that night in the Cave of Treasures.  And that night, while he was praying, Satan appeared unto him under the figure of a man, who said to him.

\begin{dialogue}
	\speak{Satan}"Thou hast oftentimes moved thy father to make an offering, to fast and to pray, therefore I will kill thee, and make thee perish from this world."
\end{dialogue}

In the face of this formidable threat, Abel's unwavering determination and steadfast faith prevailed. Strengthened by his resolute spirit, he turned to God in prayer, seeking divine intervention to combat the presence of Satan. Through his fervent supplication, Abel invoked the power of God, and the malevolent entity was forced to retreat, unable to withstand the force of his faith.

As the new day dawned, signaling their departure from the sacred Cave of Treasures, an angel of God materialized before Abel. The angel's radiant presence conveyed a sense of awe and reverence, serving as a testament to the divine intervention that had just unfolded, and said:

\begin{dialogue}
	\speak{Angel}"Shorten neither fasting, prayer, nor offering up an oblation unto thy God. For, lo, the Lord has accepted thy prayer. Be not afraid of the figure which appeared unto thee in the night, and who cursed thee unto death."
\end{dialogue}

The angel departed, leaving my brother Abel to ponder the profound encounter. Filled with a sense of urgency and concern, he approached our parents and shared the details of the vision bestowed upon him. As Mom and Dad listened to his account, their hearts were heavy with sorrow, yet they chose to remain silent, offering solace and comfort to Abel during this trying time.

Uncertainty clouded my understanding of my own role and that of my brother in these extraordinary events. I could not fathom why Abel had chosen to share his vision with me, nor did I fully grasp the significance of it all. However, that same night, I too experienced a visitation.

Under the cover of darkness, an entity approached me, assuming the voice and appearance of a person. It became evident that this figure sought to sow seeds of discord and distrust between my brother and me. His words were laced with manipulation, aiming to exploit any vulnerability within our bond.

\begin{dialogue}
	\speak{Satan}"Since Adam and Eve love thy brother Abel much more than they love thee, and wish to join him in marriage to thy beautiful sister, because they love him; but wish to join thee in marriage to his ill-favoured sister, because they hate thee; Now, therefore, I counsel thee, when they do that, to kill thy brother; then thy sister will be left for thee; and his	sister will be cast away."
\end{dialogue}

That moment proved to be a turning point for me. It became clear that Satan, the father of lies and the instigator of distrust and discord, had left my presence, yet not without planting a seed of darkness within me. Though I never harbored hatred towards my brother, the purity of my feelings for him had been tainted. The fault did not lie with him, but rather with our parents who inadvertently played a role in this situation. Despite their prayers, the connection I once had with the Divine, the One above us all, had been partially severed.

Oh, dear brother, you who embodied kindness and faithfully followed the teachings of our parents, learning from them the ways of offering and prayer. I watched and observed every time we ventured beyond the cave, witnessing the rituals performed at the altar of offerings. There, our parents would place their offerings, waiting for the divine flame to consume them, and the ascending smoke carrying their prayers to the Highest One. I nurtured the expectation and desire to emulate my brother's devotion, yet I couldn't shake the realization that we were inherently different. Despite the obligation to partake in the same rituals, I felt no genuine inclination to do so.

\section*{The offer}
The time arrived for me to present my offering, but unlike the customary practice, I made a different choice. I selected the smallest sheep from my father's flock, as I did not want to deprive him of his best and diminish his future offerings. Placing the sheep upon the altar, a realization washed over me—I understood that this innocent creature would have to be sacrificed and consumed by the flames. The weight of this realization permeated my entire being.

However, my offering was met with divine rejection. The words of God resonated in my heart, revealing that my gift was unacceptable. It became apparent that my mind had been tainted by thoughts of murder and envy, casting a shadow over the purity of my intentions. The connection between my offering and the Divine was severed, and I was made aware of this profound truth within me.

The divine rejection of my offering plunged me into sadness. I grappled with the paradox of a grand cosmic design that required the taking of life, the sacred act of burning, while expecting me to detach myself from the inherent value of that life. The weight of this realization bore down on my heart, and I couldn't help but weep. I felt a profound sense of loneliness and abandonment, despite my parents' growing concern and their increased attention towards me, surpassing that which they bestowed upon my brother.

In the ensuing years, we continued to dwell in the cave of our birth. As time passed, I reached the age of 15, while my brother reached the age of 12. It was during this period that discussions surrounding marriage began to surface.

\section*{The arranged marriage}
Indeed, an amendment to clarify the family dynamics is in order. We were all siblings, and it was revealed that Aklemia and Abel were twins. As for our betrothals, I was promised to marry Aklemia, while Abel was pledged to wed Luluwa, my sister who was born just a year after me. Despite the expectations and arrangements, I had learned to love all of my siblings deeply. Therefore, any stories or insinuations of hatred between us are false and unfounded.

One day, while I was wandering amidst the fields surrounding our cave, marveling at the tall grass and tending to the sheep, an unusual sight befell me. I caught sight of a man in the distance. This was highly uncommon, as we were the only known human beings, walking bipeds with an upright posture, at that time. The identity of this individual, whether an angel of the Lord or an agent of the Enemy, remains a mystery. Can you venture a guess as to who it might have been?

\begin{dialogue}
	\speak{Satan}"Behold, Adam and Eve have conspired to arrange your marriage with Abel's sister and his with yours."
	
	\medskip
	\direct{"Tell me something I don't already know, for I dwell in a cave, and there are no secrets there." Despite my retort, he persisted in speaking.}
	
	\speak{Satan}"But let me share a secret with you, out of love I speak. If you heed my advice, I shall bring you splendid robes, and an abundance of gold and silver on your wedding day. My kin will be in attendance, honoring you with their presence."
	
	\medskip		
	\direct{With newfound confidence, I thought I had all the arguments to defeat him and proudly questioned}	
	\speak{Cain}"Where can I find your kin?"
	\speak{Satan}"My kin dwell in a garden in the north, where once I intended to bring your father Adam. Alas, he declined my offer. But you, if you accept my proposal and come to me after your wedding, shall find respite from the misery that plagues you. You will be far better off than your father Adam."
	
	\direct{In that moment, the enemy's words further kindled the seeds of darkness planted within me before. My very being yearned to fiercely combat the evil infiltrating my heart.}	
	
\end{dialogue}

I approached my mother with a gentle demeanor, seeking understanding regarding the arrangement of our marriages. With tenderness, she guided me away from the confines of the cave and led me back to the familiar field from whence I had come. Her comforting embrace, enveloping me in her loving arms, eased the turmoil within my heart. In that moment, I realized the absurdity of the tales that spoke of me harming my mother. Such notions were baseless and contrary to the profound love and reverence I held for her.

\section*{A second offer}
As the time for new offerings arrived, my father approached me with a suggestion. He advised me to choose a young, healthy sheep with a gentle countenance as my offering to God, while my brother Abel would present an offering of corn. Together, we made our way to the sacred altar. However, the lingering emotions from the previous rejection of my offering tainted my love and admiration for my brother.

As Abel began to place his offering upon the altar, an unseen force abruptly pushed him aside, preventing him from completing his act. I was taken aback by this unexpected occurrence. What power could exert such influence, and for what purpose? Why was my beloved brother subjected to this disturbance?

Unperturbed by the interruption, Abel resumed his actions, positioning his offering on the stones adjacent to the altar. With a gesture of kindness, he stepped aside, allowing me to place my offering beside his own. Both of us patiently awaited the arrival of the sacred divine flame to consume our offerings. However, only Abel's corn was enveloped by the mesmerizing dance of the celestial fire. Its sweet and delicate fragrance filled the air, akin to the gentle scent of morning bees. God was pleased, Abel was content, but I alone felt the weight of rejection, drifting in a sea of loss and orphanhood. These emotions clung to me, refusing to dissipate. And then, from the heavens above, God spoke to me with a voice that was simultaneously profound and tender.

\begin{dialogue}
	\speak{GOD}“Why are you so upset? Why has your face fallen? Is it not thus: If you intend good, bear-it-aloft, but if you do not intend good, at the entrance is sin, a crouching-demon, toward you his lust—but you can rule over him.”
\end{dialogue}

The meaning behind God's words eluded me, leaving me uncertain and perplexed. I wondered if His message was intended as encouragement or a veiled threat. It felt like a riddle, one that I couldn't decipher. Despite my deep affection for Abel and my recognition of his virtues, I found myself at a crossroads. How could I improve my offerings when I had no better means to do so? My brother had the advantage of selecting the finest animals to sacrifice, while I remained confined by my circumstances. The nature of the sin referred to by God remained elusive, adding to my confusion.

With a countenance pale and filled with sorrow, I departed from the altar, my skin untouched by the sun's rays. I returned to my parents, bearing the weight of disappointment, and shared with them the events that had unfolded. My father's face reflected sadness, understanding the pain I felt as I faithfully followed his instructions, only to face rejection. In that moment, I longed for nothing more than to embrace my father, seeking solace in his comforting arms and sharing a glimmer of hope.

Meanwhile, Abel left the altar with a light heart, brimming with joy and love. I witnessed his happiness before I could fully immerse myself in the embrace of my father. He approached us, exuberantly proclaiming God's acceptance of his offering, seemingly oblivious to the emotions I was experiencing and the connection I was sharing with our father. His interruption was met with a subtle nod from my father, which I could discern through the slight movement of his head as he embraced me. As for my mother, she celebrated the moment with restrained jubilation.

Deep down, I couldn't help but suspect that Abel's words were not truly his own. The influence of someone familiar, the one known for sowing seeds of discord and deceit, seemed evident in his demeanor and the impact of his words.

\begin{dialogue}
	\speak{Abel}"Because Cain pushed me away from the altar and prevented me from offering my gift, I built an altar for myself and presented my offering upon it."
	
	\medskip
	\direct{Although saddened by my brother's words, I remained composed and responded as I left my father's embrace.}
	
	\speak{Cain}"Indeed, I did feel a strong desire to push you, but I restrained myself. I gripped my right arm tightly with my left, refusing to act on those impulses. Yet, somehow, you were still driven away by an unseen force. Our relationship with our grandpa is blessed and filled with joy, but ours seems to be eroding like sand gradually wearing away stone."
	
	\speak{Abel}"Brother, this seems unlikely. As time goes on, I've felt a growing sense of distrust, envy, and animosity between us."
	
	\speak{Cain}"Brother, it is the enemy who is poisoning our hearts! Why do you doubt my intentions and actions towards you? I have never committed an evil act. Do not kill me, do not extinguish the light within my heart. My spirit is tormented, struggling to resist the temptations planted by the enemy. My will and strength may not be as steadfast as yours, nor do they shine as brightly, but deep down, I yearn to be like you. Please, do not nourish the serpent within me, for it will strike the hunter who tirelessly pursues it, wearing him down."	
\end{dialogue}	

As I ventured through the field, doubts filled my mind. Was there a preference from our grandfather? If so, wouldn't that be unjust? Why did Abel, so full of unwavering faith, falter in his sensitivity during our last encounter? Could it be the sinister work of evil? Perhaps our differing visions of the world and values tilted the scales in his favor. Some situations are negotiable, but when it comes to dealing with God, it often feels like irreconcilability is the only option. With these thoughts swirling in my head, I decided to distance myself from the presence of my mother, father, and brother.

As I walked, a figure appeared before me. A tall man, draped in a golden robe, radiating brilliance from behind. His luminous presence was captivating, for he was none other than Satan, the bearer of light.

\begin{dialogue}
	\speak{Satan}"While we speak, your brother basks in the warmth of your father's embrace, your mother's tender affection, and the laughter of your sisters. They shower him with love and rejoice in his presence, all because of your perceived act of pushing him from the altar."
	
	\medskip
	\direct{It was true, I didn't push my brother. But somehow, in that moment, the truth seemed elusive, overshadowed by the relentless assault on my mind and soul. Anger and a thirst for vengeance consumed me, driving me to the brink of madness.}	
\end{dialogue}	

I loved him deeply, and I made every effort to reconcile our differences, to truly understand him. I believed that through understanding, I could find solace and overcome the torment that plagued my mind. However, something had changed within me. Sinful thoughts started to invade my consciousness, thoughts that I had never entertained before. Despite this, I was aware of my brother's well-being, and I knew his whereabouts.

The obsession to please God consumed me. It wasn't a matter of simply fulfilling my duty to offer sacrifices; it was driven by the desperate need to have my offerings accepted. Each rejection felt like a blow to my soul. This relentless pursuit of appeasement began to take its toll, wearing me down day by day.

\section*{The dream}
I found myself slipping into sleep much earlier than usual, and soon I found myself trapped in a nightmare. In the dream, my brother and I were standing before the altar, offering our sacrifices, when suddenly, blood began to rain down from the sky. The heavens darkened with ominous grey clouds, and thunderous roars filled the air, accompanied by blinding flashes of lightning. I awoke abruptly, my body drenched in sweat, my heart racing, and my vision blurred. I sensed a presence lingering, but I couldn't discern its identity as it vanished into the murky landscape. Overwhelmed by the intensity of the experience, I vomited and huddled, burying my head between my knees. Even the simplest act of breathing felt like a daunting challenge. In that moment, I couldn't comprehend why tears streamed down my face.

The dawn of a new day arrived, and it was time for the offerings once again. However, I couldn't shake the uneasiness that lingered around my interactions with Abel. I decided to approach him and engage in a heartfelt conversation, cautiously avoiding any mention of the unsettling dream I had experienced. With my staff in hand, prepared to ward off any potential threats from wild beasts, we embarked on a walk through the fields. As we strolled side by side, our conversation unfolded.

\begin{dialogue}
	\speak{CAIN}“I have come to realize that the world was created in goodness, but its governance does not align with the outcome of good deeds. It seems that favoritism plays a role in judgment. That is why your offering was accepted, and mine was not received with the same goodwill.”
	
	\speak{ABEL}“"Indeed, the world was created in goodness, and it is governed based on the fruits of good works. There is no partiality in judgment. It is simply because the fruits of my labor were superior to yours, and they were presented earlier. That is why my offering was accepted with goodwill.”
\end{dialogue}

I was filled with intense frustration and disbelief. How could he be so blind to the realities of the world? How could he naively trust in someone who would undoubtedly favor him? Was he completely oblivious to the concept of bias? My emotions boiled over, and I couldn't hold back the torrent of fiery words that erupted from within me.

\begin{dialogue}
	\speak{CAIN}I believe there is no judgment, no Judge, no other world. There will be no reward for the righteous or punishment for the wicked.”
	
	\medskip
	\direct{My brother again replied in a monothonic voice that was as flat as a lake with no waves or winds on it's surface}	
	\speak{ABEL} “On the contrary, there is indeed a judgment, a Judge, and another world. The righteous will receive a good reward, and the wicked will face vengeance.”
\end{dialogue}

\section*{The sins}
My patience dwindled to nothingness as we stood alone in the desolate field, devoid of life except for the whispering wind and the rustling of the wheat. The absence of sheep and birds only intensified my frustration. I was consumed by a desperate desire to dismantle his arguments, but the words eluded me, exacerbating my anguish. With each passing moment, my distress grew exponentially, and I succumbed to the dominance of my unruly passions and emotions. It was an obsession that gripped me, a lust for unraveling the enigma that God had presented before me.

In the grip of this relentless pursuit, the weight of God's words vanished from my heart, and self-concern became my sole focus. I became consumed by avarice, a greed that fixated on solving the riddle. How could my brother, in his unwavering faith, display such indifference towards me? If God bestowed favor upon him, it meant that he and God shared common values, including a lack of empathy for others—particularly me. I couldn't help but envy him, resenting the bond they shared. \textbf{Envy}.

My brother interrupted my thought saying.
\begin{dialogue}
	\speak{Abel}"Behold, my brother, we are weary of walking ; for we see none of the trees, nor of the fruits, nor of the verdure, nor of the sheep, nor any one of the things of which thou didst tell me. Where are those sheep of thine thou didst tell me to bless?"
	\speak{Cain}"Come on, and presently thou shalt see many beautiful things, but go before me, until I come up to thee."
\end{dialogue}

And so we trudged forward, Abel leading the way, and I begrudgingly following in his footsteps. My mind churned with thoughts of my brother's selfishness, how he shamelessly claimed moments that rightfully belonged to me and my father. It was a despicable trait, one that I believed would earn God's disapproval. Lost in my contemplation, my gaze fixated on my staff, a symbol of my power and authority, when suddenly, my brother's voice shattered my reverie. It was in that instant, as his interruption cut through me, that an overwhelming surge of wrath consumed my being.  \textbf{Wrath}.

\begin{dialogue}
	\speak{Abel}"O, my brother, have pity on me. By the breasts we have sucked, smite me not ! By the womb that bare us and that brought us into the world, smite me not unto death with that staff ! If thou wilt kill me, take one of these large stones, and kill me outright."
\end{dialogue}

Overwhelmed by a storm of emotions, I questioned how my brother could possess such insight, delving into the depths of my mind and soul to perceive my intentions. Did God grant him this ability? His perfection and purity remained unscathed even on the precipice of the abyss, speaking to me with an unwavering calmness as if we were engaged in the most mundane of tasks. All the sins that had taken root within me surged forth, obscuring my ability to perceive the truth of God, causing His Word to vanish from my heart. My vision blurred, and an insidious Beast seized control, leaving little room for restraint. Only one action remained at my disposal, obedient to the bidding of my brother—I reached for a sizeable rock, abandoning my staff, and struck his head with a force that released the contents of his brain. In that moment, a twisted sense of relief washed over me.

However, as soon as my heinous act was complete, my mother and father arrived on the scene, their eyes falling upon the grim sight of my brother's lifeless body at my feet. Overwhelmed with shame and guilt, I could not bear to meet their gaze, my eyes averted. The stone slipped from my trembling hands, landing upon my brother's head once more, causing a disturbing involuntary movement.

\section*{The consequences}
Blood flowed freely, staining the earth crimson, while ominous clouds swallowed the expanse of the sky. A haunting chorus of celestial voices resounded, their lament echoing through the depths of my soul. As the heavens crackled with electric energy, a bolt of lightning struck perilously close to where I stood, the crackling power resonating through my very being. And in that moment, amidst the raging tempest, God's voice pierced through with an eerie calmness.

\begin{dialogue}
	\speak{God}"Where is Abel thy brother?"
	\speak{Cain}"How, O God ? Am I my brother's keeper?"
	
	\medskip
	\direct{God spoke louder and His voice was filled with angry and sorrow}	
	
	\speak{God}"Cursed be the earth that has drunk the blood of Abel thy brother; and thou, be thou shaking and shaking; and this will be a sign together with you, that whosoever finds thee, shall kill thee."
	\speak{Cain}"O God, whosoever finds me shall kill me, and I shall be blotted out from the face of the earth."
	\speak{God}"Whosoever shall find thee shall not kill thee; I shall forego seven punishments on him who kills Cain."
\end{dialogue}	

I found myself engulfed in a sea of failure, unable to comprehend God's desires, my devotion crumbling, and sin devouring my soul. The mark of my transgression was etched upon me, a constant reminder of my severed connection with the divine. Desperate, I poured out my prayers, though they seemed to vanish into emptiness, leaving me drowning in a sea of hopelessness. My supplications were no longer for myself but for my brother, for I had lost faith in my own redemption, knowing the gravity of my deeds and the impending consequences.

Within me, a dark storm raged, fueled by the forces of lust, greed, wrath, envy, and pride. Wrath, in particular, tore at the fabric of my being, consuming the dwindling fragments of my once virtuous self. Allow me to digress for a moment, for there are gaps in the holy texts that recount this tale. I do not venture to speculate on the reasons behind these omissions, but I caution against taking sides in the absence of complete information. In her book "The Seven Deadly Sins: A Companion," Dorothy L. Sayers aptly described wrath as the distortion of love for justice, transformed into revenge and spite. This description resonated deeply within me, for I had become a vessel of wrath, driven by a thirst for retribution.

Yet, let me assure you that I am no longer that person. Please bear with me as I recount the untold events that unfolded in the aftermath, shedding light on the missing pieces of this harrowing tale.

\begin{center}
	\includegraphics[scale=0.52]{art/images/cain/wandering-1}
	\label{fig:lilith-01}
\end{center}